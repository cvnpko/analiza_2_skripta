\documentclass{article}

\usepackage[T1]{fontenc}
\usepackage[utf8]{inputenc}
\usepackage{graphicx}
\usepackage{makecell}
\usepackage{amssymb}
\usepackage{amsmath}
\usepackage{amsfonts}
\usepackage{tcolorbox}
\usepackage{enumitem}
\usepackage{microtype}
\usepackage{tikz}
\usepackage[colorlinks=true, allcolors=blue]{hyperref}
\usepackage[a4paper,top=2cm,bottom=2cm,left=2cm,right=2cm]{geometry}

\newtcolorbox{defbox}{
    colback = blue!10!white,
    colframe = blue,
    boxrule = 0.5pt,
    arc = 4pt,
    boxsep = 3pt
}

\newtcolorbox{teoremabox}{
    colback = red!30!white,
    colframe = red,
    boxrule = 0.5pt,
    arc = 4pt,
    boxsep = 3pt
}

\newtcolorbox{algbox}{
    colback = red!30!white,
    colframe = red,
    boxrule = 0.5pt,
    arc = 4pt,
    boxsep = 3pt
}

\newtcolorbox{stavbox}{
    colback = orange!15!white,
    colframe = orange,
    boxrule = 0.5pt,
    arc = 4pt,
    boxsep = 3pt
}

\newtcolorbox{lemabox}{
    colback = yellow!30!white,
    colframe = yellow,
    boxrule = 0.5pt,
    arc = 4pt,
    boxsep = 3pt
}

\newtcolorbox{tvrbox}{
    colback = green!10!white,
    colframe = green,
    boxrule = 0.5pt,
    arc = 4pt,
    boxsep = 3pt
}

\newtcolorbox{primbox}{
    colback = green!10!white,
    colframe = green,
    boxrule = 0.5pt,
    arc = 4pt,
    boxsep = 3pt
}

\newtheorem{definicija}{Definicija}[section]
\newtheorem{teorema}{Teorema}[section]
\newtheorem{stav}{Stav}[section]
\newtheorem{lema}{Lema}[section]
\newtheorem{tvr}{Tvrđenje}[section]
\newtheorem{alg}{Algoritam}[section]
\newtheorem{prim}{Primer}[section]

\DeclareMathOperator{\tg}{tg}
\DeclareMathOperator{\ctg}{ctg}
\DeclareMathOperator{\arcctg}{arcctg}
\DeclareMathOperator{\arctg}{arctg}

\begin{document}

\begin{titlepage}

    \newcommand{\HRule}{\rule{\linewidth}{0.4mm}}
    \center
    \textsc{\LARGE Matematički fakultet}\\[7cm]

    \HRule\\[0.4cm]
    {\LARGE\bfseries Odgovori na teorijska ispitna pitanja iz analize 2}
    \\[0.2cm]
    \HRule\\[2cm]

    \vspace{20\baselineskip}
    \begin{minipage}[t]{0.4\textwidth}
        \begin{flushleft}
            \large
            \textit{Radili}\\
            Lazar Jovanović 34/2023\\
            Jana Vuković 124/2022
        \end{flushleft}
    \end{minipage}
    \hspace*{1cm}
    \begin{minipage}[t]{0.4\textwidth}
        \begin{flushright}
            \large
            \textit{Profesor}\\
            dr Marek Svetlik
        \end{flushright}
    \end{minipage}

    \vfill\vfill\vfill\vfill
    {\large Beograd, 2024/2025}
    \vfill

\end{titlepage}

\renewcommand{\contentsname}{Sadržaj}
\tableofcontents

\newpage

\section{Neodređeni integrali}
\subsection{Primitivna funkcija}

Posmatrajmo neku funkciju $f: \mathbb{R} \longrightarrow \mathbb{R}$, na primer
$f\left(x\right) = x^2$. Možemo da pronađemo koeficijent pravca u tački
$x_0 \in \mathbb{R}$ računanjem izraza $\lim\limits_{h \to 0} \frac{f\left(x_0+h\right) - f\left(x_0\right)}{h}$.
U našem primeru dobijamo
$\lim\limits_{h \to 0} \frac{\left(x_0+h\right)^2 - \left(x_0\right)^2}{h} = $
$\lim\limits_{h \to 0} 2 x_0 + h = 2 x_0$.
Dakle, koeficijent pravca funkcije $f$ u tački $x_0$ jeste broj $2 x_0$.
Na ovaj način imamo određenu novu funkciju
$\phi : \mathbb{R} \longrightarrow \mathbb{R}$ definisanu
sa $\phi\left(x\right) = 2 x$. Uobičajeno je da funkciju $\phi$ nazivamo
izvodna funkcija (izvod, prvi izvod) funkcije $f$.
Funkciju $\phi$ drugačije označavamo sa $f'$.\par
Sada razmotrimo obratan problem. Odredimo funkciju $f: \mathbb{R} \longrightarrow \mathbb{R}$,
ako je poznato da je funkcija
$f': \mathbb{R} \longrightarrow \mathbb{R}$ definisana sa
$f'\left(x\right) = 2 x$. Iz prošlog primera možemo da zaključimo da je $f\left(x\right) = x^2$ jedno rešenje. Zapitajmo
se da li je i jedino. Nije, na primer funkcija $f\left(x\right) = x^2 + 1$
je takođe rešenje.\par
Pokušajmo da odredimo funkciju $f: \mathbb{R} \longrightarrow \mathbb{R}$ ako je
poznato da je
$$
    f'\left(x\right) = \text{sgn}\left(x\right)=
    \begin{cases}
        1,\ x > 0 \\
        0,\ x = 0 \\
        -1,\ x < 0
    \end{cases}.
$$
Takvo $f$ ne postoji, jer funkcija $\text{sgn}$ ima prekid prve vrste.

\begin{teoremabox}
    \label{podsetnik_teoreme_1}
    \textbf{Podsetnik teoreme:} Neka je $f: \mathbb{R}\longrightarrow\mathbb{R}$ diferencijabilna funkcija.
    Tada funkcija $f'$ ne može imati prekide prve vrste.
\end{teoremabox}

\begin{defbox}
    \label{podsetnik_definicije_1}
    \textbf{Podsetnik definicije prekida prve vrste:} Za funkciju $f$,
    tačka $x_0$ je prekid prve vrste ako postoje konačni $\lim\limits_{x\rightarrow x_0^+}f\left(x\right)$
    i $\lim\limits_{x\rightarrow x_0^-}f\left(x\right)$ i različiti su.
\end{defbox}

\begin{defbox}
    \label{definicija_1.1}
    \begin{definicija}
        Neka je $f:\left(a, b\right) \longrightarrow \mathbb{R}$.
        Funkciju $F:\left(a, b\right) \longrightarrow \mathbb{R}$ nazivamo primitivna
        funkcija za funkciju $f$ na intervalu $\left(a, b\right)$ ako je funkcija F
        diferencijabilna na $\left(a, b\right)$ i za svako $x \in \left(a,b\right)$ važi
        $F'\left(x\right) = f\left(x\right)$.
    \end{definicija}
\end{defbox}

Prirodno se postavljaju pitanja da li za datu funkciju postoji
primitivna funkcija i ako postoji koliko primitivnih funkcija ima.
O broju primitivnih funkcija nam govori sledeći stav.

\begin{stavbox}
    \label{stav_1.1}
    \begin{stav}
        Neka je $f: \left(a, b\right) \longrightarrow \mathbb{R}$
        i neka je $F: \left(a,b\right) \longrightarrow \mathbb{R}$
        primitivna funkcija za funkciju $f$ na intervalu $\left(a, b\right)$
        i neka je $C \in \mathbb{R}$ proizvoljno. Tada je
        funkcija $G: \left(a, b\right) \longrightarrow \mathbb{R}$,
        definisana sa $G\left(x\right) = F\left(x\right) + C$, primitivna funkcija za
        funkciju f na intervalu $\left(a, b\right)$.
    \end{stav}
\end{stavbox}

\textit{Dokaz}: $G$ je diferencijabilna na $\left(a, b\right)$ jer je zbir dve
diferencijabilne funkcije i za svako $x$ iz intervala $\left(a, b\right)$ važi $G'\left(x\right) = F'\left(x\right) + 0 = f\left(x\right)$,
što smo i hteli da dokažemo.
\null\hfill $\blacksquare$\par

Ovim smo dokazali da ako je $F_1\left(x\right)$ primitivna funkcija, onda
je i $F_1\left(x\right)+C$ primitivna funkcija. Sledeće pitanje je da li
može da postoji neka funkcija $F_2\left(x\right)$ koja nije ovog oblika.
O tome nam govori sledeća teorema.

\begin{teoremabox}
    \label{teorema_1.1}
    \begin{teorema}
        Neka je $f: \left(a, b\right) \longrightarrow \mathbb{R}$ i
        neka su $F_1, F_2: \left(a,b\right) \longrightarrow \mathbb{R}$
        primitivne funkcije za funkciju $f$ na intervalu $\left(a, b\right)$.
        Tada postoji $C \in \mathbb{R}$ takvo da
        za svako $x\in\left(a, b\right)$ važi $F_1\left(x\right) = F_2\left(x\right) + C$.
    \end{teorema}
\end{teoremabox}

\textit{Dokaz}: Neka je funkcija $G: \left(a, b\right) \longrightarrow \mathbb{R}$
definisana sa $G\left(x\right) = F_1\left(x\right) - F_2\left(x\right)$. Tada važi
$$G'\left(x\right) = F'_1\left(x\right) - F'_2\left(x\right) = f\left(x\right) - f\left(x\right) = 0.$$
Izaberimo proizvoljne $x_1, x_2 \in \left(a, b\right)$ takve da važi
$x_1 < x_2$. Dokažimo da je $G\left(x_1\right) = G\left(x_2\right)$.
\begin{enumerate}[label=(\arabic*)]
    \item $G$ je neprekidna na $\left[x_1, x_2\right] \subset \left(a, b\right)$
    \item $G$ je diferencijabilna na $\left(x_1, x_2\right) \subset \left(a, b\right)$
\end{enumerate}
Iz $\left(1\right)$ i $\left(2\right)$, a na osnovu Lagranžove teoreme o srednjoj vrednosti,
sledi da postoji $x_0 \in \left(x_1, x_2\right)$ takvo da:
$$G\left(x_1\right) - G\left(x_2\right) = G'\left(x_0\right)\left(x_2-x_1\right) = 0  \left(x_2-x_1\right) = 0.$$
Dakle, $G\left(x_1\right) = G\left(x_2\right)$. Kako su $x_1$ i $x_2$
proizvoljni, sledi da je $G$ konstantna funkcija. Važi
da postoji $C\in\mathbb{R}$ takvo da za svako $x\in\left(a, b\right)$ važi
jednakost $C=G\left(x\right) =F_1\left(x\right) - F_2\left(x\right)$, odakle je $F_1\left(x\right) = F_2\left(x\right) + C$.
\null\hfill $\blacksquare$\par

\begin{teoremabox}
    \label{podsetnik_teoreme_2}
    \textbf{Podsetnik Lagranžove teoreme o srednjoj vrednosti:} Neka je funkcija
    $f : \left[a,b\right]\longrightarrow\mathbb{R}$ neprekidna na $\left[a,b\right]$
    i difernecijabilna na $\left(a,b\right)$. Tada će
    postojati tačka $x_0\in\left(a,b\right)$ takva da važi
    $$\displaystyle\frac{f\left(b\right)-f\left(a\right)}{b-a}=f'\left(x_0\right).$$
\end{teoremabox}

\begin{primbox}
    \label{primer_1.1}
    \begin{prim}
        Neka je $f: \mathbb{R} \longrightarrow \mathbb{R}$
        definisana sa: $f\left(x\right) = 2x$. Odrediti:
        \begin{enumerate}[label=\alph*)]
            \item sve primitivne funkcije za funkciju $f$.\\
                  To su funkcije $x^2 + C,\  C \in \mathbb{R}$.
            \item funkciju $g: \mathbb{R} \longrightarrow \mathbb{R}$
                  koja je primitivna za funkciju $f$ i za koju važi
                  $g\left(0\right) = \sqrt{2}$ i $g\left(x\right) = x^2 + C,\ C \in \mathbb{R}$.
                  Rešenje je $g\left(x\right) = x^2 + \sqrt{2}$.
        \end{enumerate}
    \end{prim}
\end{primbox}

\begin{primbox}
    \label{primer_1.2}
    \begin{prim}
        Odrediti sve dvaput diferencijabilne funkcije
        $f: \mathbb{R} \longrightarrow \mathbb{R}$ takve da
        za svako $x \in \mathbb{R}$ važi $f''\left(x\right) = 0$.\par
    \end{prim}
    Ako je $\left(f'\left(x\right)\right)' = 0$ tada pogađanjem dobijamo $f'\left(x\right) = C_1$, $C_1 \in \mathbb{R}$
    i     $f\left(x\right) = C_1x+C_2,\ C_2 \in \mathbb{R}$. v  cv
    Rešenje je $f\left(x\right) = C_1x + C_2,\ C_1, C_2 \in \mathbb{R}$.
\end{primbox}

\subsection{Definicija i osnovna svojstva neodređenog integrala}
\begin{defbox}
    \label{definicija_1.2}
    \begin{definicija}
        Neka je $f: \left(a, b\right) \longrightarrow \mathbb{R}$.
        Neodređeni integral funkcije $f$ na intervalu $\left(a, b\right)$ je
        skup svih primitivnih funkcija za funkciju $f$ na intervalu
        $\left(a, b\right)$. Neodređeni integral funkcije $f$ obeležavamo sa
        $\displaystyle\int f\left(x\right)\text{dx}$. Formalno
        $$\int f\left(x\right) \text{dx} = \Bigl\{F\ \Big|\ F: \left(a, b\right) \longrightarrow \mathbb{R},\
            \bigl(\forall x\in\left(a,b\right)\bigr)\bigl(F'\left(x\right) = f\left(x\right)\bigr)\Bigr\}.$$
        Neka je $F: \left(a,b\right) \longrightarrow \mathbb{R}$ proizvoljna
        primitivna funkcija za funkciju $f$ na $\left(a,b\right)$. Tada je:
        \setcounter{equation}{0}
        \begin{equation} \label{eq_1.1.1}
            \int f\left(x\right) \, \text{dx} =\Bigl\{G\ \Big|\ G: \left(a, b\right) \longrightarrow \mathbb{R},\ \left(\exists C \in \mathbb{R}\right) \bigl(\forall x \in \left(a, b\right)\bigr)\bigl(G\left(x\right) = F\left(x\right) + C\bigr)\Bigr\}
        \end{equation}
        Jednakost \eqref{eq_1.1.1} skraćeno zapisujemo na sledeći način
        \begin{equation}\label{eq_1.1.2}
            \int f\left(x\right)\text{dx} = F\left(x\right) + C,\ C\in\mathbb{R}
        \end{equation}
    \end{definicija}
    Napomena: U jednakosti \eqref{eq_1.1.2} ne vidi se interval $\left(a, b\right)$ što
    stvara potencijalnu opasnost.
\end{defbox}

\begin{primbox}
    \label{primer_1.3}
    \begin{prim}
        Neka je $n \in \mathbb{N}$ i neka su $a_n, a_{n-1},
            \dotsc, a_1, a_0 \in \mathbb{R}$. Naći integral
        $\displaystyle\int \left(a_n  x^n + \dotsc + a_0\right)\text{dx}$.
    \end{prim}
    Nagađanjem možemo da dođemo do rešenja:
    $\displaystyle\int \left(a_n  x^n + \dotsc + a_0\right)\text{dx}=\frac{a_n}{n+1}x^{n+1} + \dotsc +
        a_0x + C,\ C\in\mathbb{R}$.
\end{primbox}

\begin{primbox}
    \label{primer_1.4}
    \begin{prim}
        Naći $\displaystyle\int\frac{1}{x}\text{dx}$.
    \end{prim}
    Neka je $\displaystyle f\left(x\right) = \frac{1}{x}$.
    Nije naglašeno na kom intervalu rešavamo integral, zbog čega uzimamo domen
    funkcije
    $D_f= \mathbb{R}\backslash \left\{0\right\} = \left(-\infty, 0\right)\cup\left(0, +\infty\right)$.\par
    Imamo dva intervala pa posmatramo dva slučaja.
    \begin{enumerate}[leftmargin=2cm, label=\arabic*. slučaj:]
        \item $x \in \left(0, +\infty\right)$ $\displaystyle\int \frac{1}{x}\text{dx} =
                  \ln x + C_1=\ln|x| + C_1,\ C_1\in\mathbb{R}$.
        \item $x \in \left(-\infty, 0\right)$ $\displaystyle\int \frac{1}{x}\text{dx} =
                  \ln\left(-x\right) + C_2=\ln|x| + C_2,\ C_2\in\mathbb{R}$.
    \end{enumerate}
\end{primbox}

Bitno je naglasiti da se konstante $C_1$ i $C_2$ odnose na intervale. One
u opštem slučaju ne moraju da budu jednake.
U sledećem primeru vidimo gde može da nastane problem.

\begin{primbox}
    \label{primer_1.5}
    \begin{prim}
        Odrediti funkciju $f: \left(-\infty, 0\right)\cup\left(0, +\infty\right)
            \longrightarrow \mathbb{R}$, takvu da $f\left(1\right) = 0$,
        $f\left(-1\right) = 1$ i za svako $x \in \left(-\infty, 0\right)
            \cup\left(0, +\infty\right)$ važi
        $\displaystyle f'\left(x\right) = \frac{1}{x}$.
    \end{prim}
    \textbf{Pogrešno rešenje:} Iz \hyperref[primer_1.4]{prethodnog primera} dobijamo $\displaystyle f\left(x\right) = \int\frac{1}{x}\text{dx} = \ln|x| + C$. Ubacivanjem vrednosti
    dobijamo $f\left(1\right)  = 0 = \ln\left(1\right) + C=C$ i $f\left(-1\right) = 1 = \ln\left(1\right) + C =C$.
    Dobijamo da je $C=0=1$, što je kontradikcija.\par
    \textbf{Tačno rešenje:}\par
    Posmatrajmo $f: \left(-\infty, 0\right)\cup\left(0, +\infty\right)$
    definisanu sa
    $$
        f\left(x\right) =
        \begin{cases}
            \ln x,\ x > 0 \\
            \ln \left(-x\right) + 1,\ x < 0
        \end{cases}.
    $$
    Njen izvod je:
    $$f'\left(x\right) =
        \begin{cases}
            \frac{1}{x},\ x > 0 \\
            \frac{1}{x},\ x < 0
        \end{cases}.
    $$
    Dakle, funkcija $f$ je tražena funkcija jer važi
    $f\left(1\right) = 0$ i $f\left(-1\right) = 1$. U ovom primeru je $C_1=0$, a $C_2=1$.
\end{primbox}

\begin{stavbox}
    \label{stav_1.2}
    \begin{stav}
        Neka su $f_1, f_2 : \left(a, b\right) \longrightarrow \mathbb{R}$
        funkcije koje imaju primitivne funkcije na $\left(a, b\right)$ i neka
        su $\lambda_1 , \lambda_2 \in \mathbb{R}$. Tada funkcija
        $\lambda_1 f_1 + \lambda_2 f_2 : \left(a, b\right) \longrightarrow
            \mathbb{R}$ ima primitivnu funkciju na $\left(a, b\right)$ i važi:\par
        $$\int \left(\lambda_1 f_1 + \lambda_2 f_2\right)\left(x\right)\text{dx} = \lambda_1\int
            f_1\left(x\right) \text{dx} + \lambda_2\int f_2\left(x\right) \text{dx}.$$
    \end{stav}
\end{stavbox}

\textit{Dokaz}: Neka je funkcija $F_1: \left(a, b\right) \longrightarrow \mathbb{R}$
primitivna funkcija za $f_1$ i neka je funkcija $F_2: \left(a, b\right)
    \longrightarrow \mathbb{R}$ primitivna funkcija za $f_2$.
Tada, po definiciji primitivne funkcije, za svako $x\in\left(a,b\right)$ važe jednakosti
$F'_1\left(x\right) = f_1\left(x\right)$ i $F'_2\left(x\right) = f_2\left(x\right)$.
Odatle zaključujemo:
$$\left(\lambda_1 F_1 + \lambda_2 F_2\right)'\left(x\right) =
    \lambda_1 F'_1\left(x\right) + \lambda_2 F'_2\left(x\right) =
    \lambda_1 f_1\left(x\right) + \lambda_2 f_2\left(x\right) =
    \left(\lambda_1 f_1 + \lambda_2 f_2\right)\left(x\right).$$
Ako krenemo od desne strane jednakosti koju dokazujemo i
primenimo prethodne jednakosti dobijamo:
\begin{align*}
    \lambda_1\int f_1\left(x\right) \text{dx} + \lambda_2\int f_2\left(x\right) \text{dx} & =\lambda_1\left(F_1\left(x\right) + C_1\right) + \lambda_2\left(F_2\left(x\right) + C_2\right) \\
                                                                                          & =\lambda_1  F_1\left(x\right) + \lambda_2  F_2\left(x\right) + \lambda_1  C_1 + \lambda_2  C_2 \\
                                                                                          & =\lambda_1  F_1\left(x\right) + \lambda_2  F_2\left(x\right) + C                               \\
                                                                                          & =\left(\lambda_1 F_1 + \lambda_2 F_2\right)\left(x\right) + C                                  \\
                                                                                          & =\int \left(\lambda_1 f_1+ \lambda_2 f_2\right)\left(x\right)\text{dx},
\end{align*}
gde su $C_1,C_2\in\mathbb{R}$, a $C=\lambda_1C_1+\lambda_2C_2$. Ovim završavamo dokaz.
\null\hfill $\blacksquare$\par

\begin{primbox}
    \label{primer_1.6}
    \begin{prim}
        Naći $\displaystyle\int \left(\frac{3}{\sqrt{x}} +
            \cos\frac{x}{3} - 5\cdot 2^x\right)\text{dx},\ x > 0$.
    \end{prim}
    \begin{align*}
        I & = \int\frac{3}{\sqrt{x}} \text{dx} + \int \cos\frac{x}{3} \text{dx} - 5\int 2^x \text{dx}
        \\ & = 6\sqrt{x} + C_1 + 3\sin\frac{x}{3} + C_2 - 5\frac{2^x}{\ln{2}} - 5C_3
        \\ & = 6\sqrt{x} + 3\sin\frac{x}{3} - 5\frac{2^x}{\ln{2}} + C,
    \end{align*}
    gde su $C_1,C_2,C_3\in\mathbb{R}$, a $C=C_1+C_2-5C_3$.
\end{primbox}

\setlength{\tabcolsep}{2.55em}
\renewcommand{\arraystretch}{2}
\begin{tabular}{ |c|c| }
    \hline
    \multicolumn{2}{|c|}{Tablica integrala:}                                                                                                                                                                                                              \\ \hline
    $\displaystyle\alpha \in \mathbb{R} \backslash \left\{-1\right\},\ x\in\left(0, +\infty\right)\quad \int x^{\alpha} \text{dx}$                   & $\displaystyle\frac{1}{\alpha + 1} x^{\alpha + 1} + C$                                             \\ \hline
    $\displaystyle n \in \mathbb{N},\ x \in \mathbb{R} \quad \int x^{n} \text{dx}$                                                                   & $\displaystyle\frac{1}{n + 1} x^{n + 1} + C$                                                       \\ \hline
    $\displaystyle n \in \mathbb{N}\backslash\left\{1\right\},\ x \in\mathbb{R}\backslash\left\{0\right\}\quad \int x^{-n} \text{dx} $               &
    \makecell{
    $\displaystyle\frac{1}{1-n} x^{1-n} + C_1,\ x \in \left(-\infty, 0\right)$                                                                                                                                                                            \\
        $\displaystyle\frac{1}{1-n} x^{1-n} + C_2,\ x \in \left(0, +\infty\right)$
    }                                                                                                                                                                                                                                                     \\  \hline
    $\displaystyle x \in \mathbb{R} \quad \int x^{-1} \text{dx}$                                                                                     &
    \makecell{ $\ln |x| + C_1,\ x \in \left(-\infty, 0\right)$                                                                                                                                                                                            \\
    $\displaystyle\ln |x| + C_2,\ x \in \left(0, +\infty\right)$}                                                                                                                                                                                         \\ \hline
    $\displaystyle a > 0,\ a\neq 1,\ x \in \mathbb{R} \quad \int a^x \text{dx}$                                                                      & $ \displaystyle\frac{1}{\ln a} a^x + C$                                                            \\ \hline
    $\displaystyle x \in \mathbb{R} \quad \int \sin x \text{dx}$                                                                                     & $ \displaystyle-\cos x + C$                                                                        \\ \hline
    $\displaystyle x \in \mathbb{R} \quad \int \cos x \text{dx}$                                                                                     & $ \displaystyle\sin x + C$                                                                         \\ \hline
    $\displaystyle x \in \bigcup_{k \in \mathbb{Z}} \left(\frac{\pi}{2} + k\pi, \frac{3\pi}{2} + k\pi\right)\quad \int \frac{1}{\cos^2 x} \text{dx}$ & $ \tg x + C_k,\ x \in \left(\frac{\pi}{2} + k\pi, \frac{3\pi}{2} + k\pi\right),\ k \in \mathbb{Z}$ \\ \hline
    $\displaystyle x \in \bigcup_{k \in \mathbb{Z}} \left(k\pi, \pi + k\pi\right)\quad \int \frac{1}{\sin^2 x} \text{dx}$                            & $ \displaystyle-\ctg x + C_k,\ x \in \left(k\pi, \pi + k\pi\right),\ k \in \mathbb{Z}$             \\ \hline
    $\displaystyle x \in \left(-1, 1\right) \quad \int \frac{1}{\sqrt{\left(1 - x^2\right)}} \text{dx}                      $                        & $ \arcsin x + C$                                                                                   \\ \hline
    $\displaystyle x \in \mathbb{R} \quad \int \frac{1}{1+x^2} \text{dx}$                                                                            & $ \arctg x + C$                                                                                    \\ \hline
    $\displaystyle x \in \mathbb{R}\backslash \left\{0\right\} \quad \int x^{0} \text{dx}$                                                           &
    \makecell{
    $x + C_1,\ x \in \left(-\infty, 0\right)$                                                                                                                                                                                                             \\
        $x + C_2,\ x \in \left(0, +\infty\right)$
    }                                                                                                                                                                                                                                                     \\ \hline
\end{tabular}

\subsection{Metode integracije}

\begin{teoremabox}
    \label{teorema_1.2}
    \begin{teorema}
        (Teorema o smeni promenljive 1) Neka je $F: \left(a, b\right) \longrightarrow \mathbb{R}$ primitivna funkcija za funkciju $f:\left(a, b\right) \longrightarrow \mathbb{R}$ i neka je $g: \left(\alpha, \beta\right) \longrightarrow \left(a, b\right)$ diferencijablna funkcija na $\left(\alpha, \beta\right)$. Tada postoji primitivna funkcija za funkciju $\left(f\circ g\right) g' : \left(\alpha, \beta\right) \longrightarrow \mathbb{R}$ i važi:\par
        \begin{equation*}
            \int \left(\left(f\circ g\right) g'\right)\left(x\right)\text{dx} = F\left(g\left(x\right)\right) + C,\ C\in\mathbb{R}
        \end{equation*}
    \end{teorema}
\end{teoremabox}
\textit{Dokaz}: Neka je $x \in \left(a, b\right)$ proizvoljno. Tada
\begin{align*}
    \hspace*{1cm}    \left(F\left(g\left(x\right)\right) + C\right)' & = \left(\left(F \circ g\right)\left(x\right) + C\right)'  = \left(\left(F \circ g\right)\left(x\right)\right)' + 0 \\
                                                                     & = F'\left(g\left(x\right)\right)  g'\left(x\right)        = f\left(g\left(x\right)\right)  g'\left(x\right)        \\
                                                                     & = \left(\left(f \circ g\right)  g'\right)\left(x\right)
\end{align*}
\null\hfill $\blacksquare$\par

\begin{primbox}
    \label{primer_1.7}
    \begin{prim}
        Neka su $a \in \mathbb{R}\backslash\left\{0\right\},\ b\in\mathbb{R},\ n \in \mathbb{Z}\backslash\left\{-1\right\}$. Naći $\displaystyle\int\left(ax + b\right)^n\text{dx}$.
    \end{prim}
    Neka su $f\left(t\right) = t^n$ i $g\left(x\right) = ax+b$. Tada važi $\displaystyle F\left(t\right) = \frac{t^{n+1}}{n+1}$ i $g'\left(x\right) = a$. Primenom \hyperref[teorema_1.2]{teoreme o smeni promenljive 1} dobijamo
    \begin{align*}
        \int\left(ax + b\right)^n\text{dx} & = \int f\left(g\left(x\right)\right)\text{dx} = \int\frac{1}{a}f\left(g\left(x\right)\right)g'\left(x\right)\text{dx} \\
                                           & = \frac{1}{a}\int f\left(g\left(x\right)\right)g'\left(x\right)\text{dx}                                              \\
                                           & = \frac{1}{a}F\left(g\left(x\right)\right) + C                                                                        \\
                                           & = \frac{1}{a}\frac{1}{1+n}\left(ax + b\right)^{1+n}.
    \end{align*}
    Primitivna funkcija koju smo dobili je definisana na $x \in \mathbb{R}$ za $n > 0$.\\
    Za $n\in \mathbb{Z} \cap \left(\left(-\infty, -1\right)\cup\left\{0\right\}\right)$ je definisana na $\displaystyle x \in \left(-\infty, -\frac{b}{a}\right)\cup\left(-\frac{b}{a}, +\infty\right)$.\par
    Napomena: Izdvajamo slučaj kada je $n=0$ jer za $\displaystyle x=-\frac{b}{a}$ dobijamo $0^0$. Iako važi $\lim\limits_{x\longrightarrow0+}x^x = 1$, izraz
    $0^0$ nije definisan jer $\lim\limits_{\left(x, y\right)\longrightarrow\left(0,0\right)} x^y$ ne postoji.
\end{primbox}

\begin{primbox}
    \label{primer_1.8}
    \begin{prim}
        Naći $\displaystyle\int \frac{e^x}{\sqrt{e^x + 1}}\text{dx}$.
    \end{prim}
    Neka su $\displaystyle f\left(t\right) = \frac{1}{\sqrt{t}}$ i $g\left(x\right) = e^x+1$. Tada važi $F\left(t\right) = 2  \sqrt{t}$ i $g'\left(x\right) = e^x$. Primernom \hyperref[teorema_1.2]{teoreme o smeni promenljive 1}
    dobijamo
    \begin{align*}
        \int\frac{e^x}{\sqrt{e^x + 1}}\text{dx} & = \int f\left(g\left(x\right)\right)e^x\text{dx} = \int f\left(g\left(x\right)\right)g'\left(x\right)\text{dx} \\
                                                & = F\left(g\left(x\right)\right) + C                                                                            \\
                                                & = 2 \sqrt{ e^x+1}.
    \end{align*}
\end{primbox}

\begin{teoremabox}
    \label{teorema_1.3}
    \begin{teorema}
        (Teorema o smeni promenljive 2)
        Neka je $f:\left(a, b\right) \longrightarrow \mathbb{R}$, neka je
        $g:\left(\alpha, \beta\right) \longrightarrow \left(a, b\right)$
        diferencijabilna funkcija takva da postoji
        $g^{-1}:\left(a,b\right) \longrightarrow \left(\alpha, \beta\right)$
        koja je takođe diferencijabilna i neka je \\
        $F:\left(\alpha, \beta\right) \longrightarrow \mathbb{R}$
        primitivna funkcija za funkciju
        $\left(f\circ g\right) g' : \left(\alpha, \beta\right) \longrightarrow \mathbb{R}$.
        Tada postoji primitivna funkcija za funkciju f i važi:
        $\displaystyle\int f\left(x\right) \text{dx} = F\left(g^{-1}\left(x\right)\right) + C,\ C \in \mathbb{R}$.
    \end{teorema}
\end{teoremabox}

\textit{Dokaz}: Izaberimo proizvoljno $x \in \left(a, b\right)$. Tada
\begin{align*}
    \left(F\left(g^{-1}\left(x\right)\right) + C\right)' & = \left(F\left(g^{-1}\left(x\right)\right)\right)' + 0 = F'\left(g^{-1}\left(x\right)\right) \left(g^{-1}\right)'\left(x\right)                                                                                                                                                      \\
                                                         & = \left(\left(f\circ g\right) g'\right)\left(g^{-1}\left(x\right)\right) \left(g^{-1}\right)'\left(x\right)                                                                                                                                                                          \\
                                                         & = \left(f\circ g\right)\left(g^{-1}\left(x\right)\right) g'\left(g^{-1}\right) \left(g^{-1}\right)'\left(x\right)                                                                                                                                                                    \\
                                                         & = \left(f\left(x\right)\right) \left(g \circ g^{-1}\right)'\left(x\right)                                                        = f\left(x\right) x'                                                                                                             = f\left(x\right).
\end{align*}
\null\hfill $\blacksquare$\par

Pogledajmo sada par primera koji ilustruju kad smemo da
koristimo \hyperref[teorema_1.3]{teoremu 1.3}.\par

\begin{primbox}
    \label{primer_1.9}
    \begin{prim}
        Neka je $g: \mathbb{R} \longrightarrow \left(0, +\infty\right)$ definisano sa
        $g\left(t\right) = e^t$ i neka je $g^{-1}:\left(0, +\infty\right)\longrightarrow \mathbb{R}$
        definisano sa $g^{-1}\left(x\right) = \ln{x}$.
    \end{prim}
    Funkcije $g$ i $g^{-1}$
    su diferencijabilne na svojim domenima pa možemo primeniti teoremu.
\end{primbox}

\begin{primbox}
    \label{primer_1.10}
    \begin{prim}
        Neka je $g:\mathbb{R}\longrightarrow\mathbb{R}$
        definisano sa $g\left(t\right) = t^3$ i neka je $g^{-1}:\mathbb{R}\longrightarrow\mathbb{R}$
        definisano sa $g^{-1}\left(x\right) = \sqrt[3]{x}$.
    \end{prim}
    Funkcija $g$ je diferencijabilna
    na svom domenu. Ostaje da proverimo da li je i funkcija $g^{-1}$. Proverimo da li je ona diferencijabilna u nuli:
    $\left(g^{-1}\right)'\left(0\right) = \lim\limits_{h\longrightarrow 0}\frac{g^{-1}\left(h\right) - g^{-1}\left(0\right)}{h} = \lim\limits_{h\longrightarrow 0}\frac{^3\sqrt{h} - 0}{h} = \lim\limits_{h\longrightarrow 0}\frac{1}{^3\sqrt{h^2}} = +\infty$.
    Dakle, nije diferencijabilno, zbog čega ne može primeniti teoremu.\par
\end{primbox}

\begin{primbox}
    \label{primer_1.11}
    \begin{prim}
        Neka je $g: \left(0, +\infty\right) \longrightarrow \left(0, +\infty\right)$
        definisano sa $ g\left(t\right) = t^3$ i neka je $g^{-1}: \left(0,+\infty\right)\longrightarrow\left(0,+\infty\right)$
        definisano sa $ g^{-1}\left(x\right) = \sqrt[3]{x}$.
    \end{prim}
    Funkcija $g$ je diferencijabilna na svom domenu.
    Za izvod funkcije $g^{-1}$ dobijamo: $\left(g^{-1}\right)'\left(x\right) = \frac{1}{3}\frac{1}{\sqrt[3]{x^2}}$.
    Ovo je definisano na celom domenu pa možemo da primenimo teoremu.\par
\end{primbox}

Sledi primer korišćenja \hyperref[teorema_1.3]{teoreme 1.3}.

\begin{primbox}
    \label{primer_1.12}
    \begin{prim}
        Neka je $a > 0$. Naći $\int\sqrt{a^2 - x^2}\text{dx}$.
    \end{prim}
    Neka je $f:\left(-a, a\right)\longrightarrow(0,a]$ definisano sa $f\left(x\right) = \sqrt{a^2 - x^2}$ i neka je
    $g:\left(-\frac{\pi}{2}, \frac{\pi}{2}\right) \longrightarrow \left(-a, a\right)$ definisano sa $g\left(x\right) = a \sin x$.
    Tada je $g^{-1}:\left(-a, a\right)\longrightarrow\left(-\frac{\pi}{2}, \frac{\pi}{2}\right)$ definisano sa $g^{-1}\left(x\right) = \arcsin\frac{x}{a}$.\\
    Izvod $g'\left(x\right) = a \cos{x}$ je definisan na celom domenu pa možemo da koristimo \hyperref[teorema_1.3]{teoremu 1.3}.
    Zbog preglednijeg zapisa, neka je $g^{-1}\left(x\right)=t$.
    \begin{align*}
        \int  \sqrt{a^2 - x^2} \text{dx} & = \int f\left(x\right) \text{dx}=F\left(g^{-1}\left(x\right)\right)+C =\int \left(\left(f\circ g\right) g'\right)\left(t\right)dt = \int a\cos\left(t\right) \sqrt{a^2 - a^2\sin^2\left(t\right)} dt \\
                                         & = \int  a\cos\left(t\right)\sqrt{a^2\left(1 - \sin ^2\left(t\right)\right)} dt = \int a^2 \cos\left(t\right) \sqrt{1-\sin^2 \left(t\right)}dt = \int a^2 \cos ^2 \left(t\right) dt                   \\
                                         & =\int\frac{a^2}{2}\left(1+\cos \left(2 t\right)\right) dt= \int\frac{a^2}{2} + \frac{a^2}{2} \cos\left(2 t\right) dt= \frac{a^2}{2}t  + \frac{a^2}{4}\sin \left(2 t\right)+C                         \\
    \end{align*}
    Dobili smo rešenje integrala, ali ovo rešenje možemo još da sredimo.
    \begin{align*}
          & \frac{a^2}{2}\arcsin\frac{x}{a} + \frac{a^2}{4}\sin \left(2 \arcsin\frac{x}{a}\right)+C                                                                                         \\
        = & \frac{a^2}{2}\arcsin\frac{x}{a} + \frac{a^2}{4} 2sin\left(\arcsin\frac{x}{a}\right)cos\left(\arcsin\frac{x}{a}\right) + C & \textit{(formula za polovinu ugla sinusa)}          \\
        = & \frac{a^2}{2}\arcsin\frac{x}{a} + \frac{a^2}{2}\frac{x}{a}cos\left(\arcsin\frac{x}{a}\right) + C                          & \textit{\hyperref[napomena_1_primer_1.9]{napomena}} \\
        = & \frac{a^2}{2}\arcsin\frac{x}{a} + \frac{a x}{2}\sqrt{cos^2\left(\arcsin\frac{x}{a}\right)} + C                            & \textit{(kosinus na domenu je pozitivan)}           \\
        = & \frac{a^2}{2}\arcsin\frac{x}{a} + \frac{a x}{2}\sqrt{1-\sin^2\left(\arcsin\frac{x}{a}\right)} + C                         &                                                     \\
        = & \frac{a^2}{2}\arcsin\frac{x}{a} + \frac{a x}{2}\sqrt{1-\frac{x^2}{a^2}} + C                                               &                                                     \\
        = & \frac{a^2}{2}\arcsin\frac{x}{a} + \frac{x}{2}\sqrt{a^2-x^2} + C                                                           &                                                     \\
    \end{align*}
    \label{napomena_1_primer_1.9}Napomena: Iako važi da je $\sin\left(\arcsin\left(x\right)\right)=x$, ne mora da važi da je $\arcsin\left(\sin\left(x\right)\right)=x$.\par
    Na kraju možemo da proverimo da li smo dobili tačno rešenje tako što uradimo izvod primitivne funkcije.
\end{primbox}

Opisali smo kako se ponaša integral linearnih kombinacije funkcija i kako se uvode smene. Ostaje da vidimo
šta se dešava sa integralom proizvoda dve funkcije. O tome nam govori sledeća teorema.

\begin{teoremabox}
    \label{teorema_1.4}
    \begin{teorema}
        (Teorema o parcijalnoj integraciji) Neka su
        $u, v: \left(a, b\right) \longrightarrow \mathbb{R}$ diferencijabilne
        funkcije. Tada funkcija $u v':\left(a, b\right) \longrightarrow \mathbb{R}$
        ima primitivnu funkciju ako i samo ako funkcija $u' v: \left(a, b\right) \longrightarrow \mathbb{R}$
        ima primitvnu funkciju. Važi
        $$\int u\left(x\right)v'\left(x\right)\text{dx} = u\left(x\right)v\left(x\right) -\int u'\left(x\right)v\left(x\right)\text{dx}.$$
    \end{teorema}
\end{teoremabox}

\textit{Dokaz}: Neka je $x \in \left(a, b\right)$ proizvoljno. Tada
\begin{equation}\label{teorema_jednakost_1_4_1}
    \left(uv\right)'\left(x\right) = \bigl(u\left(x\right) v\left(x\right)\bigr)' = u'\left(x\right)v\left(x\right) + u\left(x\right)v'\left(x\right)
\end{equation}
Pretpostavimo da $u'v$ ima primitivnu funkciju. Tada iz \eqref{teorema_jednakost_1_4_1} dobijamo:
\begin{equation}\label{teorema_jednakost_1_4_2}
    u\left(x\right)v'\left(x\right) = \bigl(u\left(x\right) v\left(x\right)\bigr)' - u'\left(x\right)v\left(x\right)
\end{equation}
Kako $\bigl(u\left(x\right)v\left(x\right)\bigr)'$ i $u'\left(x\right)v\left(x\right)$ imaju primitivne funkcije,
iz \eqref{teorema_jednakost_1_4_2} sledi da je i $u\left(x\right)v'\left(x\right)$ ima. Osim toga važi:
$$\int u\left(x\right)v'\left(x\right)\text{dx} = \int\left(u\left(x\right)v\left(x\right)\right)'\text{dx} - \int u'\left(x\right)v\left(x\right)\text{dx} = u\left(x\right)v\left(x\right) -\int u'\left(x\right)v\left(x\right)\text{dx}.$$
Drugi smer dokaza je potpuno analogan.
\null\hfill $\blacksquare$\par

\begin{primbox}
    \label{primer_1.13}
    \begin{prim}
        Naći $\displaystyle\int e^x \sin x \text{dx}$.
    \end{prim}
    Primenjujemo \hyperref[teorema_1.4]{teoremu o parcijalnoj integraciji} za $u\left(x\right) = e^x$ i $v\left(x\right) = -\cos x$.
    $$\displaystyle\int e^x\sin x \text{dx} = -e^x\cos x + \int e^x \cos x\text{dx}$$
    Ostaje da se izračuna $\displaystyle\int e^x\cos x \text{dx}$. Ponovo primenjujemo \hyperref[teorema_1.4]{teoremu o parcijalnoj integraciji}, ovaj
    put za $u\left(x\right) = e^x$ i $ v\left(x\right) = \sin x$.
    $$\displaystyle\int e^x\cos x \text{dx} = e^x\sin x - \int e^x\sin x \text{dx}$$
    Uvrstimo dobijeni izraz u prethodnu jednakost.
    \begin{align*}
        \int e^x \sin x \text{dx}  & = -e^x\cos x + e^x \sin x - \int e^x \sin x \text{dx}                 \\
        2\int e^x \sin x \text{dx} & = -e^x\cos x + e^x \sin x +C_1,\ C_1\in\mathbb{R}                     \\
        \int e^x \sin x \text{dx}  & = \frac{1}{2}\left(-e^x\cos x + e^x \sin x\right)+C,\ C=\frac{C_1}{2}
    \end{align*}
\end{primbox}

\begin{primbox}
    \label{primer_1.14}
    \begin{prim}
        Naći $\displaystyle\int \frac{1}{x} \text{dx}$ za $x > 0$.
    \end{prim}
    Iskoristićemo \hyperref[teorema_1.4]{teoremu o parcijalnoj integraciji} na $\displaystyle u\left(x\right)  = \frac{1}{x}$ i
    $v\left(x\right) = x$.
    Dobijamo jednakost
    $$\int \frac{1}{x} \text{dx} = \frac{x}{x} + \int \frac{1}{x} = 1+ \int\frac{1}{x}.$$
    Greška koja se često pravi je da se integrali skrate i da se dobije $1 = 0$, što znači da ovaj integral ne postoji.
    Ovo nije tačno jer ovu jednakost možemo da zapišemo i kao $F+C_1 = 1+F+C_2,\ C_1,C_2\in\mathbb{R}$.
    Ako skratimo primitivne funkcije dobijamo vezu između konstanti, što nam nije od pomoći. Metodom pogađanja
    rešenja dobijamo $$\int\frac{1}{x}\text{dx}=\ln x+C,\ C\in\mathbb{R}.$$
\end{primbox}

\subsection{Integracija racionalnih funkcija}

Integraciju racionalnih funkcija možemo da rešavamo po algoritmu.
Potrebno je da znamo rešenja integrala polinoma i integrala
oblika
$\displaystyle\int \frac{A}{x-a}\text{dx}$, $\displaystyle\int\frac{1}{\left(x-a\right)^k}\text{dx}$,
$\displaystyle\int \frac{Mx+N}{x^2+bx + c}\text{dx}$ i $\displaystyle\int \frac{Mx + N}{\left(x^2 + bx + c\right)^k}\text{dx}$,
gde su $A,a,M,N,b,c\in\mathbb{R}$, $b^2-4c<0$, $k\in\mathbb{N}\backslash\left\{1\right\}$.
Integral polinoma već znamo, a u sledećim lemama ćemo
pokazati proces nalaženja ovih integrala.

\begin{lemabox}
    \label{lema_1.1}
    \begin{lema}
        Važi $\displaystyle\int \frac{A}{x-a}\text{dx}=A\ln|x-a| + C$, gde su $A,a,C\in \mathbb{R}$.
    \end{lema}
\end{lemabox}

\textit{Dokaz:} Neka je funkcija $f:\mathbb{R}\backslash\left\{0\right\}\longrightarrow\mathbb{R}$ definisana sa
$\displaystyle f\left(t\right)=\frac{A}{t}$ i funkcija $g:\mathbb{R}\backslash\left\{a\right\}\longrightarrow\mathbb{R}\backslash\left\{0\right\}$
definisana sa $g\left(x\right)=x-a$. Funkcija $g$ je diferencijabilna na svom domenu, gde je $g'\left(x\right)=1$, a primitivna funkcija
funkcije $f$ je funkcija $F\left(t\right)=A\ln|t|$. Tada možemo da primenimo \hyperref[teorema_1.2]{teoremu}.
$$ \int \frac{A}{x-a}\text{dx} = \int \left(\left(f\circ g\right) g'\right)\left(x\right)\text{dx} = F\left(g\left(x\right)\right) + C=A\ln|x-a| + C.$$
Napomenimo da $C$ za $x > a$ i $C$ za $x < a$ mogu biti različiti.
\null\hfill $\blacksquare$\par

\begin{lemabox}
    \label{lema_1.2}
    \begin{lema}
        Važi $\displaystyle\int \frac{A}{\left(x-a\right)^k}\text{dx}=\frac{A}{\left(1-k\right)\left(x-a\right)^{k-1}}+ C$, gde su $A,a,C\in \mathbb{R}$, $k\in\mathbb{N}\backslash\left\{1\right\}$.
    \end{lema}
\end{lemabox}

\textit{Dokaz:} Neka je funkcija $f:\mathbb{R}\backslash\left\{0\right\}\longrightarrow\mathbb{R}$ definisana sa
$f\left(t\right)=\frac{A}{t^k}$ i funkcija $g:\mathbb{R}\backslash\left\{a\right\}\longrightarrow\mathbb{R}\backslash\left\{0\right\}$
definisana sa $g\left(x\right)=x-a$. Funkcija $g$ je diferencijabilna na svom domenu, gde je $g'\left(x\right)=1$, a primitivna funkcija
funkcije $f$ je funkcija $F\left(t\right)=\frac{A}{\left(1-k\right)t^{k-1}}$. Tada možemo da primenimo \hyperref[teorema_1.2]{teoremu 1.2}.
$$\int \frac{A}{\left(x-a\right)^k}\text{dx} = \int \left(\left(f\circ g\right) g'\right)\left(x\right)\text{dx} = F\left(g\left(x\right)\right) + C= \frac{A}{\left(1-k\right)\left(x-a\right)^{k-1}}+ C.$$
\null\hfill$\blacksquare$\par

\begin{lemabox}
    \label{lema_1.3}
    \begin{lema}
        Važi $\displaystyle\int \frac{Mx+N}{x^2+bx+c}\text{dx}= \frac{M}{2}\ln\left(x^2+xb+c\right)+\frac{2N-bM}{\sqrt{4c-b^2}}\arctg\left({\frac{2x+b}{\sqrt{4c-b^2}}}\right)+ C$, gde su $M,N,b,c,C\in\mathbb{R}$, $b^2 - 4c < 0$.
    \end{lema}
\end{lemabox}

\textit{Dokaz:} Neka je funkcija $ f:\mathbb{R}\longrightarrow\mathbb{R}$ definisana sa
$\displaystyle f\left(t\right)=\frac{Mt}{t^2+1}+\frac{2N-bM}{\left(t^2+1\right)\sqrt{4c-b^2}}$ i funkcija $g:\mathbb{R}\longrightarrow\mathbb{R}$
definisana sa $\displaystyle g\left(x\right)=\frac{2x+b}{\sqrt{4c-b^2}}$. Funkcija $g$ je diferencijabilna na svom domenu, gde je $\displaystyle g'\left(x\right)=\frac{2}{\sqrt{4c-b^2}}$.
Potrebno je da izračunamo primitivnu funkciju:
\begin{align*}
    \int f\left(t\right)dt & =\int \frac{Mt}{t^2+1}+\frac{2N-bM}{\left(t^2+1\right)\sqrt{4c-b^2}}dt                        \\
                           & =\frac{M}{2}\int \frac{2t}{t^2+1}dt+\frac{2N-bM}{\sqrt{4c-b^2}}\int\frac{1}{t^2+1}dt          \\
                           & =\frac{M}{2}\ln\left(t^2+1\right)+\frac{2N-bM}{\sqrt{4c-b^2}}\arctg{t}+C_1,\ C_1\in\mathbb{R} \\
                           & =F\left(t\right)+C_1
\end{align*}
Tada možemo da primenimo \hyperref[teorema_1.2]{teoremu o smeni promenljive 1}:
\begin{align*}
    \int \frac{Mx+N}{x^2+bx+c}\text{dx} & = \int \left(\left(f\circ g\right) g'\right)\left(x\right)\text{dx}                                                                                                         \\
                                        & = F\left(g\left(x\right)\right) + C_1                                                                                                                                       \\
                                        & = \frac{M}{2}\ln\left(\left(\frac{2x+b}{\sqrt{4c-b^2}}\right)^2+1\right)+\frac{2N-bM}{\sqrt{4c-b^2}}\arctg\left({\frac{2x+b}{\sqrt{4c-b^2}}}\right)+ C_1,\ C_1\in\mathbb{R} \\
                                        & = \frac{M}{2}\ln\left(\frac{4x^2+4xb+b^2+4c}{4c-b^2}\right)+\frac{2N-bM}{\sqrt{4c-b^2}}\arctg\left({\frac{2x+b}{\sqrt{4c-b^2}}}\right)+ C_1                                 \\
                                        & = \frac{M}{2}\left(\ln\left(x^2+xb+c\right)-\ln\left(c-\frac{b^2}{4}\right)\right)+\frac{2N-bM}{\sqrt{4c-b^2}}\arctg\left({\frac{2x+b}{\sqrt{4c-b^2}}}\right)+ C_1          \\
                                        & = \frac{M}{2}\ln\left(x^2+xb+c\right)+\frac{2N-bM}{\sqrt{4c-b^2}}\arctg\left({\frac{2x+b}{\sqrt{4c-b^2}}}\right)+ C,\ C=C_1+ \frac{M}{2}\ln\left(c-\frac{b^2}{4}\right).
\end{align*}
\null\hfill $\blacksquare$\par

\begin{lemabox}
    \label{lema_1.4}
    \begin{lema}
        Važi $\displaystyle\int \frac{Mx+N}{\left(x^2+bx+c\right)^k}\text{dx}=\frac{M}{2}\frac{1}{\left(1-k\right)\left(x^2+bx+c\right)^{k-1}}+\left(N-\frac{Mb}{2}\right)I_k$, gde su $k\in\mathbb{N}\backslash\left\{1\right\}$,
        $M,N,b,c\in\mathbb{R}$, $b^2 - 4c < 0$, $\displaystyle I_k=\int\frac{1}{\left(x^2+bx+c\right)^k}\text{dx}$ i naći vezu između $I_k$ i $I_{k-1}$.
    \end{lema}
\end{lemabox}

\textit{Dokaz:} Prvo ćemo početni integral razložiti na dva lakša
\begin{align*}
    \int \frac{Mx+N}{\left(x^2+bx+c\right)^k}\text{dx} & =\int \frac{\frac{M}{2}\left(2x+b\right)-\frac{Mb}{2}+N}{\left(x^2+bx+c\right)^k}\text{dx}                                                \\
                                                       & =\int \frac{\frac{M}{2}\left(2x+b\right)}{\left(x^2+bx+c\right)^k}\text{dx}+\int\frac{N-\frac{Mb}{2}}{\left(x^2+bx+c\right)^k}\text{dx}   \\
                                                       & =\frac{M}{2}\int \frac{2x+b}{\left(x^2+bx+c\right)^k}\text{dx}+\left(N-\frac{Mb}{2}\right)\int\frac{1}{\left(x^2+bx+c\right)^k}\text{dx}.
\end{align*}
Prvi integral možemo da rešimo \hyperref[teorema_1.2]{teoremomu o smeni promenljive 1}. Neka je
funkcija $f: \mathbb{R}^{+}\longrightarrow\mathbb{R}^{+}$ definisana sa
$\displaystyle f\left(t\right)=\frac{1}{t^k}$ i neka je funkcija $g: \mathbb{R}\longrightarrow\mathbb{R}^{+}$
definisana sa $g\left(x\right)=x^2+bx+c$. Funkcija $g$ je diferencijabilna na svom domenu
i važi $g'\left(x\right)=2x+b$. Jedna od primitivnih funkcija funkcije $f$ je $\displaystyle F\left(t\right)=\frac{1}{\left(1-k\right)t^{k-1}}$.
Primenom \hyperref[teorema_1.2]{teoreme o smeni promenljive 1} dobijamo:
\begin{align*}
    \int \frac{2x+b}{\left(x^2+bx+c\right)^k}\text{dx} & =\int\left(\left(f\circ g\right)g'\right)\left(x\right)\text{dx}=F\left(g\left(x\right)\right)+C,\ C\in\mathbb{R} \\
                                                       & =\frac{1}{\left(1-k\right)\left(x^2+bx+c\right)^{k-1}}+C
\end{align*}
Za drugi integral koristimo nov metod rešavanja.
Neka je $\displaystyle I_k=\int\frac{1}{\left(x^2+bx+c\right)^k}\text{dx}$. Želimo da nađemo vezu
između $I_k$ i $I_{k-1}$ jer onda rekurzivno možemo da
rešimo integral $I_k$ (znamo rešenje $I_1$ iz \hyperref[lema_1.3]{leme 1.3}). Koristićemo \hyperref[teorema_1.4]{teoremu 1.4}.
Neka su $u,v:\mathbb{R}\longrightarrow\mathbb{R}$ definisane sa $\displaystyle u\left(x\right)=\frac{1}{\left(x^2+bx+c\right)^k}$
i $v\left(x\right)=x$. Funkcije $u$ i $v$ su diferencijabilne na svojim domenima i
važe jednakosti $\displaystyle u'\left(x\right)=-\frac{\left(2x+b\right)k}{\left(x^2+bx+c\right)^{k+1}}$ i $v'\left(x\right)=1$.
Primenom \hyperref[teorema_1.4]{teoereme o parcijalnoj integraciji} dobijamo:
\begin{align*}
    I_k                  & =\int\frac{1}{\left(x^2+bx+c\right)^k}\text{dx} =\int u\left(x\right)v'\left(x\right)\text{dx} =u\left(x\right)v\left(x\right)-\int u'\left(x\right)v\left(x\right)\text{dx} \\
                         & =\frac{x}{\left(x^2+bx+c\right)^k}+k\int\frac{2x^2+2bx+2c-bx-2c}{\left(x^2+bx+c\right)^{k+1}} \text{dx}                                                                      \\
                         & =\frac{x}{\left(x^2+bx+c\right)^k}+2k\int\frac{1}{\left(x^2+bx+c\right)^k}\text{dx}-k\int\frac{bx+2c}{\left(x^2+bx+c\right)^{k+1}} \text{dx}                                 \\
    \left(1-2k\right)I_k & =\frac{x}{\left(x^2+bx+c\right)^k}-k\int\frac{bx+2c}{\left(x^2+bx+c\right)^{k+1}} \text{dx}                                                                                  \\
                         & =\frac{x}{\left(x^2+bx+c\right)^k}-k\int\frac{\frac{b}{2}\left(2x+b\right)-\frac{b^2}{2}+2c}{\left(x^2+bx+c\right)^{k+1}} \text{dx}                                          \\
                         & =\frac{x}{\left(x^2+bx+c\right)^k}+\frac{b}{2}\frac{1}{\left(x^2+bx+c\right)^{k}}-\left(2c-\frac{b^2}{2}\right)k\int\frac{1}{\left(x^2+bx+c\right)^{k+1}} \text{dx}          \\
                         & =\frac{x+\frac{b}{2}}{\left(x^2+bx+c\right)^k}-\left(2c-\frac{b^2}{2}\right)kI_{k+1}
\end{align*}
Odredili smo vezu između $I_k$ i $I_{k+1}$, pomeranjem indeksa za jedan
naniže dobijamo vezu između $I_{k-1}$ i $I_k$
$$I_k =\frac{2x+b}{\left(4c-b^2\right)\left(x^2+bx+c\right)^k}-\frac{2-4k}{4c-b^2}I_{k-1}.$$
Na kraju, vraćanjem svega u početni integral dobijamo:
$$    \int \frac{Mx+N}{\left(x^2+bx+c\right)^k}\text{dx}=\frac{M}{2}\frac{1}{\left(1-k\right)\left(x^2+bx+c\right)^{k-1}}+\left(N-\frac{Mb}{2}\right)I_k.$$
\null\hfill $\blacksquare$\par

\begin{stavbox}
    \label{podsetnik_stava_1}
    \begin{stav}
        (Osnovni stav algebre) Svaki kompleksni polinom
        jedne promenljive stepena $n$, $n>0$, ima tačno
        $n$ kompleksnih nula.
    \end{stav}
\end{stavbox}

Dodatno, uz osnovni stav algebre važi da i ako je neki čisto imaginarni
broj jedna nula polinoma, tada je i njegov kompleksno konjugovani
par takođe nula. Zbog toga važi sledeće tvrđenje.

\begin{tvrbox}
    \label{tvrđenje_1.1}
    \begin{tvr}
        Svaki polinom $P:\mathbb{R}\longrightarrow\mathbb{R}$ stepena $n$
        može se zapisati u obliku:
        $$P\left(x\right)=p\left(x-a_1\right)^{s_1}\dotsc\left(x-a_k\right)^{s_k}\left(x-d_{11}\right)^{t_1}\left(x-d_{12}\right)^{t_1}\dotsc\left(x-d_{l1}\right)^{t_l}\left(x-d_{l2}\right)^{t_l}$$
        gde je $p$ koeficijent uz najstariji član,
        $a_i$, $i\in\left\{1\dotsc k\right\}$, realna nula polinoma $P$ mnogostrukosti $s_i$, važi $\left(\forall i_1,i_2\in\left\{1\dotsc k\right\}\right)\left(a_{i_1}=a_{i_2}\iff i_1=i_2\right)$,
        a $d_{j1}$ i $d_{j2}$,
        $j\in\left\{1\dotsc l\right\}$, konjugovano kompleksne čisto imaginarne nule polinoma $P$
        mnogostrukosti $t_j$, važi
        $$\left(\forall j_1,j_2\in\left\{1\dotsc l\right\}\right)\left(\left(d_{j_11}=d_{j_21}\iff j_1=j_2\right)\land\left(d_{j_11}\neq d_{j_22}\right)\right).$$
        Takođe, zbog \hyperref[podsetnik_stava_1]{osnovnog stava algebre}, važi jednakost $n=s_1+\dotsc+s_k+2\left(t_1+\dotsc+t_l\right)$. Dodatno,
        množenjem konjugovano kompleksnih nula dobijamo izraze oblika $x^2+bx+c$, $b,c\in{R}$, $b^2-4c<0$,
        pa izraz možemo da zapišemo u obliku
        $$P\left(x\right)=p\left(x-a_1\right)^{s_1}\dotsc\left(x-a_k\right)^{s_k}\left(x^2+b_1x+c_1\right)^{t_1}\dotsc\left(x^2+b_lx+c_l\right)^{t_l}.$$
    \end{tvr}
\end{tvrbox}

Pokazali smo kako se nalaze integrali nekih vrsta
racionalnih funkcija. Ostaje da pokažemo kako da
proizvoljnu racionalnu funkciju svedemo na
integrale ovih vrsta.

\begin{algbox}
    \begin{alg}
        Neka funkcija $st\left(P\right)$ vraća stepen polinoma.
        Neka su dati polinomi $P, Q: \mathbb{R} \longrightarrow \mathbb{R}$, pri čemu
        važi $st\left(P\right)\geq0$, $st\left(Q\right)\geq1$. Ako polinom $Q$ nema realnih nula, primitivnu funkciju
        za funkciju $\displaystyle\frac{P}{Q}$ nalazimo na intervalu $\left(-\inf, +\inf\right)$. Inače,
        ako su $a_1 < a_2 < \dotsc < a_k$ realne nule polinoma $Q$, primitivnu funkciju
        tražimo na intervalima $\left(-\inf, a_1\right),\left(a_1, a_2\right),\dotsc,\left(a_{k-1},a_k\right),\left(a_k,+\inf\right)$.
        Integral $\displaystyle\int\frac{P\left(x\right)}{Q\left(x\right)}\text{dx}$ nalazimo sledećim koracima:
    \end{alg}
    \begin{enumerate}[label=\text{K\arabic*}]
        \item\label{algoritam_1_K1}
              Ako je $st\left(P\right) < st\left(Q\right)$ primenjujemo \hyperref[algoritam_1_K2]{K2} na integral $\displaystyle\int \frac{P\left(x\right)}{Q\left(x\right)}\text{dx}$.\\
              Inače, vršimo deljenje polinoma $P$ polinomom $Q$.
              Neka je $S$ rezultat deljenja, a $Q$ ostatak pri deljenju.
              Tada važi $P\left(x\right) = S\left(x\right)Q\left(x\right) + R\left(x\right)$, $st\left(R\right)<st\left(Q\right)$.
              Zbog linearnosti integrala važi jednakost $\displaystyle\int\frac{P\left(x\right)}{Q\left(x\right)}\text{dx} = \int S\left(x\right)\text{dx} + \int\frac{R\left(x\right)}{Q\left(x\right)}\text{dx}$.
              Znamo da izračunamo integral polinoma $S\left(x\right)$, primenjujemo \hyperref[algoritam_1_K2]{K2}
              na integral $\displaystyle\int \frac{R\left(x\right)}{Q\left(x\right)}\text{dx}$.

        \item[K2:]\label{algoritam_1_K2}
              U \hyperref[tvrđenje_1.1]{tvrđenju 1.1} smo naveli da se
              svaki polinom, pa i polinom $Q$, može zapisati u obliku:
              $$Q\left(x\right) = q\left(x - a_1\right)^{s_1}\dotsc\left(x-a_k\right)^{s_k}\left(x^2 + b_1x + c_1\right)^{t_1}\dotsc\left(x^2 + b_lx + c_l\right)^{t_l}$$
              gde je $q$ koeficijent uz najstariji član,
              $a_i$, $i\in\left\{1\dotsc k\right\}$, realna nula polinoma $Q$ mnogostrukosti $s_i$, važi $\left(\forall i_1,i_2\in\left\{1\dotsc k\right\}\right)\left(a_{i_1}=a_{i_2}\iff i_1=i_2\right)$,
              a $x^2+b_jx+c_j$, $j\in\left\{1\dotsc l\right\}$,
              proizvod $\left(x-d_{j1}\right)$ i $\left(x-d_{j2}\right)$,
              gde su $d_{j1}$ i $d_{j2}$ konjugovano kompleksne čisto imaginarne nule polinoma $Q$
              mnogostrukosti $t_j$, važi $\left(\forall j_1,j_2\in\left\{1\dotsc l\right\}\right)\left(\left(d_{j_11}=d_{j_21}\iff j_1=j_2\right)\land\left(d_{j_11}\neq d_{j_22}\right)\right)$.
              Takođe, zbog \hyperref[podsetnik_stava_1]{osnovnog stava algebre}, važi jednakost $n=s_1+\dotsc+s_k+2\left(t_1+\dotsc+t_l\right)$.
              Posle faktorizacije polinoma $Q$ prelazimo na \hyperref[algoritam_1_K3]{K3}.
        \item[K3:]\label{algoritam_1_K3}
              U ovom koraku je cilj da odredimo konstante u brojiocu takve da važi jednakost:
              $$\frac{P\left(x\right)}{Q\left(x\right)}=\sum_{i = 1}^k\left(\frac{A_{i1}}{\left(x-a_i\right)}+\dotsc+\frac{A_{is_i}}{\left(x-a_i\right)^{s_i}}\right) + \sum_{j = 1}^l\left( \frac{M_{j1}x + N_{j1}}{\left(x^2+b_jx+c_j\right)}+\dotsc+\frac{M_{jt_j}x + N_{jt_j}}{\left(x^2 + b_jx + c_j\right)^{t_j}}\right)$$
              To postižemo rešavanjem linearnih jednačina koje
              nastaju svođenjem svih sabiraka na isti imenilac.
              Nakon nalaženja ovih konstanti prelazimo na \hyperref[algoritam_1_K4]{K4}.
        \item[K4:]\label{algoritam_1_K4}
              Određujemo integral zbira koji smo dobili u prethodnom koraku.
              Svi sabirci su istih oblika kao i integrali koji su navedeni u lemama
              \hyperref[lema_1.1]{1}, \hyperref[lema_1.2]{2}, \hyperref[lema_1.3]{3} i \hyperref[lema_1.4]{4}.

    \end{enumerate}
\end{algbox}

\subsection{Integracija trigonometrijskih funkcija}

Oznaku $R\left(u, v\right)$ koristimo za predstavljanje
racionalnih funkcija sa argumenatima $u$ i $v$. Na primer, neka je $\displaystyle R\left(u, v\right) = \frac{u+v}{u-v+2}$.
Integrale oblika $\int R\left(\sin{x},\cos{x}\right)\text{dx}$ rešavamo smenom
$t = \displaystyle\tg\frac{x}{2}$, definisanom svuda sem u $x\in\left\{\left(2l+1\right)\pi\ |\ l\in\mathbb{Z}\right\}$.
Formalno, neka je funkcija $f: \mathbb{R}\longrightarrow \mathbb{R}$ definisana sa $f\left(x\right)=R\left(\sin{x},\cos{x}\right)$ i
neka je funkcija $g_0:\mathbb{R}\longrightarrow\left(-\pi,\pi\right)$
definisana sa $g_0\left(x\right)=2\arctg{x}$. Funkcija $g_0$ je bijektivna
pa postoji inverz $g_0^{-1}: \left(-\pi,\pi\right)\longrightarrow\mathbb{R}$ definisan
sa $g_0^{-1}\left(x\right)=\tg{\frac{x}{2}}$. Važi i da su $g_0$ i $g_0^{-1}$ diferencijabilne.
Potrebno je još da odredimo $\left(f\circ g_0\right)\left(x\right)$. Važe jednakosti
$\left(f\circ g_0\right)\left(x\right)=f\left(g_0\left(x\right)\right)=R\left(\sin\left(g_0\left(x\right)\right),\cos\left(g_0\left(x\right)\right)\right)=R\left(\sin\left(2\arctg\left(x\right)\right), \cos\left(2\arctg\left(x\right)\right)\right)$. Ostaje da redukujemo izraze $\sin\left(2\arctg\left(x\right)\right)$ i $\cos\left(2\arctg\left(x\right)\right)$:
\begin{align*}
    \sin\left(2\arctg\left(x\right)\right) & =2\sin\left(\arctg\left(x\right)\right)\cos\left(\arctg\left(x\right)\right)=2\tg\left(\arctg\left(x\right)\right)\cos^2\left(\arctg\left(x\right)\right)                                                                 \\
                                           & =\frac{2x}{\frac{\sin^2\left(x\right)+\cos^2\left(x\right)}{\cos^2\left(x\right)}}=\frac{2x}{\tg^2\left(\arctg\left(x\right)\right)+1} =\frac{2x}{x^2+1}                                                                  \\
    \cos\left(2\arctg\left(x\right)\right) & =\cos^2\left(\arctg\left(x\right)\right)-\sin^2\left(\arctg\left(x\right)\right)=\cos^2\left(\arctg\left(x\right)\right)\left(1-\tg^2\left(\arctg\left(x\right)\right)\right)                                             \\
                                           & =\frac{1-x^2}{\frac{\sin^2\left(\arctg\left(x\right)\right)+\cos^2\left(\arctg\left(x\right)\right)}{\cos^2\left(\arctg\left(x\right)\right)}}=\frac{1-x^2}{\tg^2\left(\arctg\left(x\right)\right)+1}=\frac{1-x^2}{x^2+1}
\end{align*}
Odavde dobijamo $\displaystyle\left(f\circ g_0\right)\left(x\right)=R\left(\frac{2x}{1+x^2},\frac{1-x^2}{1+x^2}\right)$.\par
Možemo da odredimo integral $\displaystyle\int\left(\left(f\circ g_0\right)g_0'\right)\left(x\right)\text{dx}=\int R\left(\frac{2x}{1+x^2},\frac{1-x^2}{1+x^2}\right)\frac{2}{1+x^2}\text{dx}=F\left(x\right)+C_0,\ C_0\in\mathbb{R}$,
jer je to integral racionalne funkcije. Primenom \hyperref[teorema_1.3]{teoreme 1.3} dobijamo
$\displaystyle\int f\left(x\right)\text{dx}=F\left(g_0^{-1}\left(x\right)\right)+C_0$.\par
Rešili smo integral na intervalu $\left(-\pi,\pi\right)$. Posmatrajmo intervale
oblika $\left(\left(2m-1\right)\pi,\left(2m+1\right)\pi\right),\ m\in\mathbb{Z}$. Funkcija $g_m\left(x\right)=2\arctg\left(x\right)+2m\pi$
je diferencijabilna i ima inverz $g_m^{-1}\left(x\right)=\tg\left(\frac{x+2m\pi}{2}\right)=\tg\left(\frac{x}{2}\right)$. Analogno kao za
$g_0$ dobijamo $\left(f\circ g_m\right)\left(x\right)=R\left(\sin\left(2\arctg\left(x\right)+2m\pi\right),\cos\left(2\arctg\left(x\right)+2m\pi\right)\right)=R\left(\frac{2x}{1+x^2},\frac{1-x^2}{1+x^2}\right)$.\par
Dakle, analogno rešavanju integrala na intervalu $\left(-\pi,\pi\right)$ dobijamo  $\displaystyle\int f\left(x\right)\text{dx}=F\left(g_m^{-1}\left(x\right)\right)+C_m$.\par
Rešili smo integrale na $\mathbb{R}\backslash\left\{\left(2l+1\right)\pi\ |\ l\in\mathbb{Z}\right\}$. Zbog neprekidnosti mora da važi da
je $$\left(\forall l\in\mathbb{Z}\right)\left(\lim\limits_{x\to \left(2l+1\right)\pi^{-}}F\left(g_l^{-1}\left(x\right)\right)+C_{l}=\lim\limits_{x\to \left(2l+1\right)\pi^{+}}F\left(g_{l+1}^{-1}\left(x\right)\right)+C_{l+1}\right).$$
Nalaženjem veze između konstanti $C_l$ i $C_{l+1}$ rešavamo integral i u tačkama $\left(2l+1\right)\pi$, gde je $l$ ceo broj, čime smo rešili integral
na celom $\mathbb{R}$.\par
Napomenimo da izvođenje zavisi od konkretne funkcije $f$. Za $f$ koje nije definisano na celom $\mathbb{R}$ ćemo raditi presek $\mathbb{R}$ sa $D_f$ i
posmatrati drugačije intervale, ali će postupak ostati analogan.

\begin{primbox}
    \label{primer_1.15}
    \begin{prim}
        Odrediti integral $\displaystyle\int\frac{1}{2\sin{x}-\cos{x}+5}\text{dx}$.
    \end{prim}
    Neka je $\displaystyle f\left(x\right)=\frac{1}{2\sin{x}-\cos{x}+5}$. Domen funkcije je $\mathbb{R}$. Primenjujemo prethodnu diskusiju
    $$\left(\forall l\in\mathbb{R}\right)\ \left(f\circ g_l\right)\left(x\right)=R\left(\frac{2x}{1+x^2},\frac{1-x^2}{1+x^2}\right)= \frac{1}{\frac{4x}{1+x^2}-\frac{1-x^2}{1+x^2}+5}=\frac{1+x^2}{6x^2+4x+4}$$
    Ima $l$ integrala oblika:
    \begin{align*}
        \int \left(\left(f\circ g_l\right)g_l'\right)\left(x\right)\text{dx} & =\int\frac{1+x^2}{6x^2+4x+4}\frac{2}{1+x^2}\text{dx}=\int \frac{\text{dx}}{3\left(x^2+2\frac{1}{3}x+\frac{1}{9}+\frac{5}{9}\right)}=\int\frac{\text{dx}}{3\left(x+\frac{1}{3}\right)^2+\frac{5}{3}} \\
                                                                             & =\frac{3}{5}\int\frac{\text{dx}}{\left(\frac{3}{\sqrt{5}}x+\frac{1}{\sqrt{5}}\right)^2+1}=\frac{1}{\sqrt{5}}\arctg\left(\frac{3}{\sqrt{5}}x+\frac{1}{\sqrt{5}}\right)+C_l
    \end{align*}
    Dobijamo da je $I_l =\frac{1}{\sqrt{5}}\arctg\left(\frac{3}{\sqrt{5}}\tg\left(\frac{x}{2}\right)+\frac{1}{\sqrt{5}}\right)+C_l$. Ostaje da nađemo leve i desne limese u nedefinisanim tačkama.
    \begin{align*}
        \lim\limits_{x\to \left(2l+1\right)\pi^{-}}\frac{1}{\sqrt{5}}\arctg\left(\frac{3}{\sqrt{5}}\tg\left(\frac{x}{2}\right)+\frac{1}{\sqrt{5}}\right)+C_l & =\lim\limits_{x\to \left(2l+1\right)\pi^{+}}\frac{1}{\sqrt{5}}\arctg\left(\frac{3}{\sqrt{5}}\tg\left(\frac{x}{2}\right)+\frac{1}{\sqrt{5}}\right)+C_{l+1} \\
        \frac{\pi}{2\sqrt{5}}+C_l                                                                                                                            & =-\frac{\pi}{2\sqrt{5}}+C_{l+1}                                                                                                                           \\
        C_{l+1}                                                                                                                                              & =C_l+\frac{\pi}{\sqrt{5}}
    \end{align*}
    Tada je $\displaystyle C_l=\frac{l\pi}{\sqrt{5}}+C_0$. Važi da je $\left(2l-1\right)\pi<x<\left(2l+1\right)\pi$, odakle je
    $\displaystyle\frac{x-\pi}{2\pi}<l<\frac{x+\pi}{2\pi}$, $\displaystyle l=\left[\frac{x+\pi}{2\pi}\right]$.
    Dobijamo da je rešenje integrala
    $$\int f\left(x\right)\text{dx}=\frac{1}{\sqrt{5}}\arctg\left(\frac{3}{\sqrt{5}}\tg\left(\frac{x}{2}\right)+\frac{1}{\sqrt{5}}\right)+\left[\frac{x+\pi}{2\pi}\right]\frac{\pi}{\sqrt{5}}+C_0$$
\end{primbox}

\section{Određeni integrali}

\subsection{Definicija određenog integrala}

Neka je $f:\left[a,b\right]\longrightarrow\mathbb{R}$. Želimo da definišemo $\displaystyle\int_{a}^{b}f\left(x\right)\text{dx}$.\par

\begin{defbox}
    \label{definicija_2.1}
    \begin{definicija}
        Neka su $x_0,x_1,\dotsc,x_n\in\left[a,b\right]$ tačke takve da važi $a=x_0<x_1<\dotsc<x_n=b$. Tada:
        \begin{itemize}
            \item skup intervala $\mathcal{P}=\left\{\left[x_0,x_1\right],\dotsc,\left[x_{n-1},x_n\right]\right\}$ nazivamo \textbf{podela} intervala $\left[a,b\right]$;
            \item tačke $x_0,x_1,\dotsc,x_n$ nazivamo \textbf{podeone tačke} podele $\mathcal{P}$;
            \item sa $\varDelta x_i=x_i-x_{i-1}$, gde je $i\in\left\{1,\dotsc n\right\}$, obeležavamo dužinu intervala $\left[x_{i-1},x_i\right]$;
            \item skup svih podela intervala $\left[a,b\right]$ obeležavamo sa $\mathcal{P}\left[a,b\right]$;
            \item funkciju $\lambda: \mathcal{P}\left[a,b\right]\longrightarrow \mathbb{R}$ definisanu sa $\lambda\left(\mathcal{P}\right)=\underset{i \in \left\{1, \dots, n\right\}}{\max}\varDelta x_i$ nazivamo \textbf{parametar} podele $\mathcal{P}$;
            \item za podele $\mathcal{P},\mathcal{P}'\in\mathcal{P}\left[a,b\right]$ kažemo da je podela $\mathcal{P}'$ finija od podele $\mathcal{P}$, odnosno da
                  je podela $\mathcal{P}$ grublja od podele $\mathcal{P}'$ ako je skup podeonih tačaka podele $\mathcal{P}$ podskup skupa
                  podeonih tačaka podele $\mathcal{P}'$;
            \item uređenu $n$-torku $\xi=\left\{\xi_1,\dotsc\xi_n\right\}$, $\xi_i\in\left[x_{i-1},x_i\right]$, $i\in\left\{1,\dotsc,n\right\}$ nazivamo \textbf{istaknute tačke} podele $\mathcal{P}$;
            \item uređen par $\left(\mathcal{P},\xi\right)$ nazivamo \textbf{podela sa istaknutim tačkama} intervala $\left[a,b\right]$.
        \end{itemize}
    \end{definicija}
\end{defbox}

\begin{defbox}
    \label{definicija_2.2}
    \begin{definicija}
        Neka je $f:\left[a,b\right]\longrightarrow\mathbb{R}$ i $\left(\mathcal{P},\xi\right)$ podela sa istaknutim tačkama intervala $\left[a,b\right]$. Tada zbir
        $\displaystyle\sigma\left(f,\mathcal{P},\xi\right)=\sum_{i=1}^{n}f\left(\xi_i\right)\varDelta x_i$ nazivamo \textbf{integralna suma} funkije $f$ za datu podelu
        sa istaknutim tačkama $\left(\mathcal{P},\xi\right)$ intervala $\left[a,b\right]$.
    \end{definicija}
\end{defbox}

\begin{defbox}
    \label{definicija_2.3}
    \begin{definicija}
        Ako za $f: \left[a,b\right]\longrightarrow\mathbb{R}$ i $I\in\mathbb{R}$ važi:
        $$\left(\forall\varepsilon>0\right)\left(\exists\delta>0\right)\left(\forall\left(\mathcal{P},\xi\right)\big|\mathcal{P}\in\mathcal{P}\left[a,b\right]\right)\ \lambda\left(\mathcal{P}\right)<\delta\implies|\sigma\left(f,\mathcal{P},\xi\right)-I|<\varepsilon$$
        tada kažemo da je $I$ \textbf{Rimanov integral funkcije $f$} na intervalu $\left[a,b\right]$ i pišemo:
        $$I=\lim\limits_{\lambda\left(\mathcal{P}\right)\rightarrow 0}\sigma\left(f,\mathcal{P},\xi\right)=\int_{a}^{b}f\left(x\right)\text{dx}$$
        Rimanov integral nazivamo i \textbf{određeni integral}.
    \end{definicija}
\end{defbox}

Ako za funkciju $f:\left[a,b\right]\longrightarrow\mathbb{R}$ postoji Rimanov integral, kažemo da je funkcija $f$ \textbf{R-integrabilna} na intervalu $\left[a,b\right]$.

\subsubsection{Uraditi}

\begin{primbox}
    \label{primer_2.1}
    \begin{prim}
        Ispitati da li je funkcija $f: \left[a,b\right]\longrightarrow\mathbb{R}$, definisanu sa $f\left(x\right)=c$, R-integrabilna i ako postoji, odrediti njen određeni integral.
    \end{prim}
    Uzmimo proizvoljnu podelu $\mathcal{P}$ i proizvoljan skup istaknutih tačaka $\xi$ za podelu $\mathcal{P}$. Tada važi
    \begin{equation}
        \label{primer_2.1:eq1}
        \sigma\left(f,\mathcal{P},\xi\right)=\sum_{i=1}^{n} f\left(\xi_i\right)\Delta x_i=\sum_{i=1}^{n} c\Delta x_i=c\left(b-a\right)
    \end{equation}
    Za svako $\varepsilon>0$ važi $\left|\sigma\left(f,\mathcal{P},\xi\right)-c(b-a)\right|=0<\epsilon$, dakle možemo da
    izaberemo proizvoljno $\delta$ i važi
    $\lim\limits_{\lambda\left(\mathcal{P}\right)\rightarrow0} \sigma\left(f,\mathcal{P},\xi\right)=c(b-a)$.

\end{primbox}

\begin{primbox}
    \label{primer_2.2}
    \begin{prim}
        Ispitati da li je funkcija $f: \left[a,b\right]\longrightarrow\mathbb{R}$, definisanu sa $f\left(x\right)=
            \begin{cases}
                1, & x \in \mathbb{Q} \cap \left[a,b\right]                                   \\
                0, & x \in \left(\mathbb{R} \setminus \mathbb{Q}\right) \cap \left[a,b\right]
            \end{cases}$, R-integrabilna i ako postoji, odrediti njen određeni integral.
    \end{prim}
    Neka je $\mathcal{P}$ proizvoljna podela intervala $\left[a,b\right]$. Neka je $\xi'$ skup
    istaknutih tačaka, takav da je za svako $i\in\left\{1,\dotsc,n\right\}$ $\xi_i'\in\mathbb{Q}$.
    Tada je
    \begin{equation}
        \label{primer_2.2:eq1}
        \sigma\left(f,\mathcal{P},\xi\right)=\sum_{i=1}^{n}f\left(\xi_i'\right)\Delta x_i=b-a
    \end{equation}
    Neka je $\xi''$ skup istaknutih tačaka, takav da je za svako $i\in\left\{1,\dotsc,n\right\}$ $\xi_i''\in\mathbb{R}\backslash\mathbb{Q}$.
    Tada je
    \begin{equation}
        \label{primer_2.2:eq2}
        \sigma\left(f,\mathcal{P},\xi\right)=\sum_{i=1}^{n}f\left(\xi_i''\right)\Delta x_i=0
    \end{equation}

\end{primbox}
\begin{primbox}
    \label{primer_2.3}
    \begin{prim}
        $f\left(x\right)=
            \begin{cases}
                \frac{1}{\sqrt{x}}, & x \in (0,1] \\
                0,                  & x =0
            \end{cases}$
    \end{prim}
    Negacija definicije Rimanovog integrala za $f:\left[a,b\right]\longrightarrow\mathbb{R}$ i $I\in\mathbb{R}$:
    $$\left(\exists\varepsilon>0\right)\left(\forall\delta>0\right)\left(\exists\left(\mathcal{P},\xi\right)\big|\mathcal{P}\in\mathcal{P}\left[a,b\right]\right)\ \lambda\left(\mathcal{P}\right)<\delta\land|\sigma\left(f,\mathcal{P},\xi\right)-I|\geq\varepsilon$$
\end{primbox}

\begin{teoremabox}
    \label{teorema_2.1}
    \begin{teorema}
        Ako $f: \left[a,b\right]\longrightarrow\mathbb{R}$ nije ograničena na $\left[a,b\right]$, onda ona nije R-integrabilna.
    \end{teorema}
\end{teoremabox}

\begin{defbox}
    \label{definicija_2.4}
    \begin{definicija}
        Neka je $f:\left[a,b\right]\longrightarrow\mathbb{R}$ ograničena na svom intervalu i neka je $\mathcal{P}$ podela
        intervala. Za $i\in\left\{1,\dotsc,n\right\}$ definišemo $m_i=inf\left\{f\left(x\right)|x\in\left[x_{i-1},x_i\right]\right\}$ i $M_i=sup\left\{f\left(x\right)|x\in\left[x_{i-1},x_i\right]\right\}$.
        Tada:
        \begin{itemize}
            \item zbir $\displaystyle s\left(f,\mathcal{P}\right)=\sum_{i=1}^{n}m_i\Delta x_i$ nazivamo \textbf{donja Darbuova suma} funkcije $f$ za datu podelu $\mathcal{P}$ intervala $\left[a,b\right]$;
            \item zbir $S\left(f,\mathcal{P}\right)=\sum_{i=1}^{n}M_i\Delta x_i$ nazivamo \textbf{gornja Darbuova suma} funkcije $f$ za datu podelu $\mathcal{P}$ intervala $\left[a,b\right]$.
        \end{itemize}
    \end{definicija}
\end{defbox}

\begin{defbox}
    \label{definicija_2.5}
    \begin{definicija}
        Za svake dve podele $\mathcal{P}', \mathcal{P}''\in\mathcal{P}\left[a,b\right]$ postoji finija
        podela $\mathcal{P}\in\mathcal{P}\left[a,b\right]$. Ako je skup podeonih tačaka podele $\mathcal{P}$
        unija podeonih tačaka podela $\mathcal{P}'$ i $\mathcal{P}''$, tada se podela $\mathcal{P}$
        naziva \textbf{superpozicija} podela $\mathcal{P}'$ i $\mathcal{P}''$.
    \end{definicija}
\end{defbox}

\begin{stavbox}
    \label{stav_2.1}
    \begin{stav}
        \label{Stav_2.1}
        Dokazati da za $f: \left[a,b\right]\longrightarrow\mathbb{R}$ i
        $\mathcal{P},\mathcal{P'}\in \mathcal{P}\left[a,b\right]$, gde je skup podeonih tačaka podele $\mathcal{P}$ podskup
        skupa podeonih tačaka podele $\mathcal{P}'$, važi
        $$s\left(f,P\right)\leq s\left(f,P'\right)\leq S\left(f,P'\right)\leq S\left(f,P\right).$$
    \end{stav}
\end{stavbox}

\textit{Dokaz:} Dovoljno je dokazati slučaj kad je skup podeonih tačaka podele $\mathcal{P}$
$\left\{x_0,\dotsc,x_n\right\}$, a skup podeonih tačaka podele $\mathcal{P}'$
$\left\{x_0,\dotsc,x_{k-1},x',x_k,\dotsc,x_n\right\}$. Neka je $\displaystyle s'=\sum_{i=1}^{k-1}m_i\Delta x_i+\sum_{i=k+1}^{n}m_i\Delta x_i$ i neka su
$m_k'=inf\left\{f\left(x\right)\big|x\in\left[x_{k-1},x'\right]\right\}$ i $m_k''=inf\left\{f\left(x\right)\big|x\in\left[x',x_k\right]\right\}$.
Tada je $$s\left(f,\mathcal{P}\right)=s'+m_k\left(x_k-x_{k-1}\right)$$
i
$$s\left(f,\mathcal{P}'\right)=s'+m_k'\left(x'-x_{k-1}\right)+x_k''\left(x_k-x'\right).$$
Važi da je $m_k=min\left(m_k',m_k''\right)$, odakle dobijamo:
\begin{align*}
    m_k\left(x_k-x'\right)+m_k\left(x'-x_{k-1}\right) & \leq m_k''\left(x_k-x'\right)+m_k'\left(x'-x_{k-1}\right)    \\
    s'+m_k\left(x_k-x_{k-1}\right)                    & \leq s'+m_k''\left(x_k-x'\right)+m_k'\left(x'-x_{k-1}\right) \\
    s\left(f,P\right)                                 & \leq s\left(f,P'\right).
\end{align*}
Treća nejednakost se slično dokazuje.
\null\hfill $\blacksquare$\par

\begin{stavbox}
    \label{stav_2.2}
    \begin{stav}
        Neka su $\mathcal{P}',\mathcal{P}''\in\mathcal{P}\left[a,b\right]$. Tada je $s\left(f,\mathcal{P}'\right)\leq S\left(f,\mathcal{P}''\right)$.
    \end{stav}
\end{stavbox}

\textit{Dokaz:} Neka je $\mathcal{P}$ superpozicija podela $\mathcal{P}'$ i $\mathcal{P}''$. Iz
\hyperref[stav_2.1]{prethodnog stava} važi da je
$$s\left(f,P'\right)\leq s\left(f,P\right)\leq S\left(f,P\right)$$
i da je
$$s\left(f,P\right)\leq S\left(f,P\right)\leq S\left(f,P''\right)$$
odakle dobijamo
$$s\left(f,P'\right)\leq S\left(f,P''\right).$$
Za $f: \left[a,b\right]\longrightarrow\mathbb{R}$ skup $\left\{s\left(f,P\right)\big|\mathcal{P}\in\mathcal{P}\left[a,b\right]\right\}$
je neprazan i prema \hyperref[stav_2.2]{prethodnom stavu} ograničen odozgo brojem $S\left(f,\mathcal{P}\right)$ za proizvoljno $\mathcal{P}\in\mathcal{P}\left[a,b\right]$,
pa prema aksiomi o egzistenciji supremuma postoji supremum ovog
skupa.
\null\hfill $\blacksquare$\par

\begin{defbox}
    \label{definicija_2.6}
    \begin{definicija}
        \textbf{Donji Rimanov integral} funkcije $f: \left[a,b\right]\longrightarrow\mathbb{R}$ jeste
        $$\underline{\int_{a}^{b}}f\left(x\right)\text{dx}=sup\left\{s\left(f,P\right)\big|\mathcal{P}\in\mathcal{P}\left[a,b\right]\right\}.$$
    \end{definicija}
\end{defbox}

Za $f: \left[a,b\right]\longrightarrow\mathbb{R}$ skup $\left\{S\left(f,P\right)\big|\mathcal{P}\in\mathcal{P}\left[a,b\right]\right\}$
je neprazan i prema \hyperref[stav_2.2]{prethodnom stavu} ograničen odozdo brojem $s\left(f,\mathcal{P}\right)$ za proizvoljno $\mathcal{P}\in\mathcal{P}\left[a,b\right]$,
pa prema teoremi o egzistenciji infimuma postoji infimum ovog
skupa.

\begin{defbox}
    \label{definicija_2.7}
    \begin{definicija}
        \textbf{Gornji Rimanov integral} funkcije $f: \left[a,b\right]\longrightarrow\mathbb{R}$ jeste
        $$\overline{\int_{a}^{b}}f\left(x\right)\text{dx}=inf\left\{S\left(f,P\right)\big|\mathcal{P}\in\mathcal{P}\left[a,b\right]\right\}.$$
    \end{definicija}
\end{defbox}

Jasno je da važi $\displaystyle\underline{\int_{a}^{b}}f\left(x\right)\text{dx}\leq\overline{\int_{a}^{b}}f\left(x\right)\text{dx}$.

\begin{defbox}
    \label{definicija_2.8}
    \begin{definicija}
        Ako važi da je $\displaystyle\underline{\int_{a}^{b}}f\left(x\right)\text{dx}=\overline{\int_{a}^{b}}f\left(x\right)\text{dx}=I$,
        onda broj $I$ nazivamo \textbf{Rimanov integral} funkcije $f$ na $\left[a,b\right]$ i pišemo:
        $$I=\int_{a}^{b}f\left(x\right)\text{dx}.$$
    \end{definicija}
\end{defbox}

\begin{teoremabox}
    \label{teorema_2.2}
    \begin{teorema}
        \hyperref[definicija_2.3]{Epsilon-delta definicija Rimanovog integrala} i \hyperref[definicija_2.8]{definicija Rimanovog integrala preko gornjeg i donjeg Rimanovog integrala}
        su ekvivalentne.
    \end{teorema}
\end{teoremabox}

\begin{stavbox}
    \label{stav_2.3}
    \begin{stav}
        Neka je $f: \left[a,b\right]\longrightarrow\mathbb{R}$ ograničena. Tada je funkcija $f$
        R-integrabilna na $\left[a,b\right]$ ako i samo ako za svako $\varepsilon>0$ postoji
        $\mathcal{P}\in\mathcal{P}\left[a,b\right]$ takvo da je $S\left(f,\mathcal{P}\right)-s\left(f,\mathcal{P}\right)<\varepsilon$.
    \end{stav}
    \subsubsection{Dokazati}
\end{stavbox}

\subsection{Integrabilnost nekih klasa funkcija}
\begin{teoremabox}
    \label{teorema_2.3}
    \begin{teorema}
        Neka je $f: \left[a, b\right] \longrightarrow \mathbb{R}$ neprekidna funckija na $\left[a, b\right]$. Tada je funkcija f Riman integrabilna na $\left[a, b\right]$.
    \end{teorema}
\end{teoremabox}

\begin{defbox}
    \label{podsetnik_definicije_2}
    \textbf{Podsetnik definicije monotonosti funkcije:} Funkcija $f:\left[a,b\right]\longrightarrow\mathbb{R}$
    je monotono rastuća ako važi
    $$\left(\forall x_1,x_2\in\left[a,b\right]\right)\left(x_1\leq x_2\right)\implies \left(f\left(x_1\right)\leq f\left(x_2\right)\right),$$
    a monotono opadajuća ako važi
    $$\left(\forall x_1,x_2\in\left[a,b\right]\right)\left(x_1\leq x_2\right)\implies \left(f\left(x_1\right)\geq f\left(x_2\right)\right),$$
\end{defbox}

\begin{teoremabox}
    \label{teorema_2.4}
    \begin{teorema}
        Neka je $f: \left[a, b\right] \longrightarrow \mathbb{R}$ monotona funkcija na $\left[a, b\right]$. Tada je funkcija Riman integrabilna na $\left[a, b\right]$.
    \end{teorema}
\end{teoremabox}

\textit{Dokaz:} Bez umanjenja opštosti, pretpostavimo da je funkcija monotono rastuća na $\left[a, b\right]$.
Primetimo najpre da je funkcija ograničena na $\left[a,b\right]$. Zaista, za svako $x \in \left[a, b\right]$ važi $f\left(a\right) \leq f\left(x\right) \leq f\left(b\right)$.
Pretpostavimo da funkcija nije konstantna, za konstantne funkcije već znamo da su R-integrabilne.
Neka je $\varepsilon > 0$ proizvoljno i neka je podela $\mathcal{P} \in \mathcal{P}\left[a, b\right]$ takva da je parametar podele
$\lambda\left(\mathcal{P}\right) < \frac{\varepsilon}{f\left(b\right) - f\left(a\right)}$. Tada je
\begin{align*}
    S\left(f, P\right) - s\left(f, P\right) & = \sum^n_{i = 1}\sup_{x\in\left[x_{i-1}, x_i\right]}f\left(x\right)\left(x_i - x_{i-1}\right) - \sum^n_{i = 1}\inf_{x\in\left[x_{i-1}, x_i\right]}f\left(x\right)\left(x_i - x_{i-1}\right)                                      \\
                                            & = \sum_{i=1}^n\left(f\left(x_i\right)-f\left(x_{i-1}\right)\right)\left(x_i - x_{i-1}\right) < \sum^n_{i=1}\left(f\left(x_i\right) - f\left(x_{i-1}\right)\right)\frac{\varepsilon}{f\left(b\right) - f\left(a\right)}           \\
                                            & = \frac{\varepsilon}{f\left(b\right) - f\left(a\right)}\sum^n_{i=1}\left(f\left(x_i\right) - f\left(x_{i-1}\right)\right) = \frac{\varepsilon}{f\left(b\right) - f\left(a\right)} \left(f\left(b\right) - f\left(a\right)\right) \\
                                            & = \varepsilon.
\end{align*}
Dakle, važi $S\left(f, P\right) - s\left(f, P\right)<\varepsilon$ za proizvoljno $\varepsilon$, pa po \hyperref[stav_2.3]{stavu 2.3} važi da je $f$
R-integrabilna na $\left[a,b\right]$.
\null\hfill $\blacksquare$\par

\begin{teoremabox}
    \label{teorema_2.5}
    \begin{teorema}
        Neka je $f:\left[a, b\right] \longrightarrow \mathbb{R}$ ograničena funkcija na $\left[a, b\right]$ takva da je skup tačaka u kojima nije neprekidna konačan. Tada je funkcija f Riman integrabilna na $\left[a, b\right]$.
    \end{teorema}
\end{teoremabox}

\begin{teoremabox}
    \label{teorema_2.6}
    \begin{teorema}
        Neka su $f, g: \left[a, b\right] \longrightarrow \mathbb{R}$ R-integrabilne funkcije na $\left[a, b\right]$ koje se razlikuju samo u konačno mnogo tačaka. Tada je:
        $$\displaystyle \int_a^b f\left(x\right)\text{d}x = \int_a^b g\left(x\right)\text{d}x.$$
    \end{teorema}
\end{teoremabox}

\begin{defbox}
    \label{definicija_2.9}
    \begin{definicija}
        Skup svih Riman integrabilnih funkcija na intervalu $\left[a,b\right]$ označavamo sa $R\left[a, b\right]$.
    \end{definicija}
\end{defbox}

\subsection{Svojstva određenog integrala}

\begin{stavbox}
    \label{stav_2.4}
    \begin{stav}
        Neka su $f, g \in R\left[a, b\right]$ i neka su $\alpha, \beta \in \mathbb{R}$. Tada
        $\alpha f +\beta g \in R\left[a, b\right]$ i važi:\\
        $\displaystyle \int^b_a\left(\alpha f + \beta g\right)\left(x\right)\text{dx} = \alpha\int^b_af\left(x\right)\text{dx} + \beta\int^b_ag\left(x\right)\text{dx}$
    \end{stav}
\end{stavbox}

\textit{Dokaz:} Neka je $\left(\mathcal{P}, \xi\right)$ proizvoljna podela sa istaknutim tačkama na $\left[a, b\right]$. Tada
\begin{align*}
    \sigma\left(\alpha f+\beta g, \mathcal{P}, \xi\right) & = \sum^n_{i=1} \left(x_i - x_{i-1}\right)\left(\alpha f+\beta g\right)\left(\xi_i\right)                           = \sum^n_{i=1} \left(x_i - x_{i-1}\right)\left(\left(\alpha f\right)\left(\xi_i\right) + \left(\beta g\right)\left(\xi_1\right)\right) \\
                                                          & = \sum^n_{i=1} \left(x_i - x_{i-1}\right)\left(\alpha f\right)\left(\xi_i\right) + \sum^n_{i=1} \left(x_i - x_{i-1}\right)\left(\beta g\right)\left(\xi_1\right)                                                                                          \\
                                                          & = \alpha\sum^n_{i=1} \left(x_i - x_{i-1}\right) f\left(\xi_i\right) + \beta\sum^n_{i=1} \left(x_i - x_{i-1}\right)g\left(\xi_1\right)                                                                                                                     \\
                                                          & = \alpha\sigma\left(f, \mathcal{P}, \xi\right) + \beta\sigma\left(g, \mathcal{P}, \xi\right)
\end{align*}
Po definiciji važi $\displaystyle\lim\limits_{\lambda\left(\mathcal{P}\right)\longrightarrow0} \sigma\left(f, \mathcal{P}, \xi\right) = \int^b_a f\left(x\right)\text{dx}$ i $\displaystyle \lim\limits_{\lambda\left(\mathcal{P}\right)\longrightarrow0} \sigma\left(g, \mathcal{P}, \xi\right) = \int^b_a g\left(x\right)\text{dx}$. Tada
\begin{align*}
    \int_{a}^{b}\left(\alpha f+\beta g\right)\left(x\right)\text{dx} & =\lim\limits_{\lambda\left(\mathcal{P}\right)\longrightarrow0} \sigma\left(\alpha f+\beta g, \mathcal{P}, \xi\right)                                                                                                  \\
                                                                     & =\lim\limits_{\lambda\left(\mathcal{P}\right)\longrightarrow0}\left(\alpha\sigma\left(f, \mathcal{P}, \xi\right) + \beta\sigma\left(g, \mathcal{P}, \xi\right)\right)                                                 \\
                                                                     & =\lim\limits_{\lambda\left(\mathcal{P}\right)\longrightarrow0}\alpha\sigma\left(f, \mathcal{P}, \xi\right) + \lim\limits_{\lambda\left(\mathcal{P}\right)\longrightarrow0}\beta\sigma\left(g, \mathcal{P}, \xi\right) \\
                                                                     & =\alpha\lim\limits_{\lambda\left(\mathcal{P}\right)\longrightarrow0}\sigma\left(f, \mathcal{P}, \xi\right) + \beta\lim\limits_{\lambda\left(\mathcal{P}\right)\longrightarrow0}\sigma\left(g, \mathcal{P}, \xi\right) \\
                                                                     & =\alpha \int^b_a f\left(x\right)\text{dx} +\beta\int^b_a g\left(x\right)\text{dx}.
\end{align*}
\null\hfill $\blacksquare$\par

\begin{stavbox}
    \label{stav_2.5}
    \begin{stav}
        Neka su $f, g \in R\left[a, b\right]$. Tada:
        \begin{itemize}
            \item ako je $fg \in R\left[a, b\right]$, ne mora da važi $\displaystyle \int^b_a \left(fg\right)\left(x\right)\text{dx} = \int^b_af\left(x\right)\text{dx}  \int^b_ag\left(x\right)\text{dx}$;
            \item ako je $|f|\in R\left[a, b\right]$, ne mora da važi $\displaystyle \int^b_a \big|f\left(x\right)\big|\text{dx} = \bigg|\int^b_a f\left(x\right)\text{dx}\bigg| $;
            \item $\frac{1}{f} \in R\left[a, b\right]$, pod pretpostavkom da postoji $c > 0$ takvo da za svako $x \in \left[a, b\right]$ važi $|f\left(x\right)| > c$.
        \end{itemize}
    \end{stav}
\end{stavbox}

\begin{stavbox}
    \label{stav_2.6}
    \begin{stav}
        Neka je $f\in R\left[a, b\right]$ i $\left[c, d\right] \subset \left[a, b\right]$. Tada $f\in R\left[c,d\right]$.
    \end{stav}
\end{stavbox}

\begin{stavbox}
    \label{stav_2.7}
    \begin{stav}
        Neka je $f:\left[a,b\right] \longrightarrow \mathbb{R}$ i $c \in \left(a, b\right)$. Tada $f\in R\left[a,b\right]$ ako i samo ako $f \in R\left[a, c\right]$ i $f\in R\left[c, b\right]$ i važi
        $\displaystyle\int^b_a f\left(x\right)\text{dx} = \int^c_a f\left(x\right)\text{dx} + \int^b_c f\left(x\right)\text{dx}$.
    \end{stav}
\end{stavbox}

\begin{defbox}
    \label{definicija_2.10}
    \begin{definicija}
        Neka je $f:\left\{a\right\} \longrightarrow \mathbb{R}$. Tada
        $\displaystyle\int^a_a f\left(x\right)\text{dx} \overset{\text{def}}{=} 0$.
    \end{definicija}
\end{defbox}

\begin{defbox}
    \label{definicija_2.11}
    \begin{definicija}
        Neka je $f \in R\left[a, b\right]$. Tada $\displaystyle\int^a_b f\left(x\right)\text{dx} \overset{\text{def}}{=} -\int^b_a f\left(x\right)\text{dx}$.
    \end{definicija}
\end{defbox}

\begin{defbox}
    \label{definicija_2.12}
    \begin{definicija} Neka su $a, b, c \in \mathbb{R}$ i $f:\left[\min\left\{a, b, c\right\}, \max\left\{a, b, c\right\}\right] \longrightarrow \mathbb{R}$. Tada je
        $f$ Riman integrabilna na $\left[\min\left\{a, b, c\right\}, \max\left\{a, b, c\right\}\right]$ i važi
        $\displaystyle \int^b_a f\left(x\right)\text{dx} = \int^c_a f\left(x\right)\text{dx} + \int^b_a f\left(x\right) \text{dx}$.
    \end{definicija}
\end{defbox}

\begin{stavbox}
    \label{stav_2.8}
    \begin{stav}
        Neka je $f \in R\left[a,b\right]$ i $f\left(x\right) \geq 0$, za svako $x \in \left[a,b\right]$. Tada je $\displaystyle \int^b_a f\left(x\right)\text{dx} \geq 0$.
    \end{stav}
\end{stavbox}

\textit{Dokaz:} Neka $\left(\mathcal{P}, \xi\right)$ proizvoljna podela sa istaknutim tačkama na
$\left[a, b\right]$. Tada je
$$\sigma\left(f, \mathcal{P}, \xi\right) = \displaystyle \sum^n_{i=1}f\left(\xi_i\right)\Delta x_i \geq 0.$$
Otuda je $\displaystyle \lim\limits_{\lambda\left(\mathcal{P}\right)\longrightarrow 0} \sigma \left(f, \mathcal{P}, \xi\right) \geq 0$, tačnije, $\displaystyle \int^b_a f\left(x\right)\text{dx} \geq 0$.
\null\hfill $\blacksquare$\par

\begin{stavbox}
    \label{stav_2.9}
    \begin{stav}
        Neka je $f \in R\left[a, b\right]$. Tada je $\displaystyle \bigg|\int^b_a f\left(x\right)\text{dx}\bigg| \leq \int^b_a\big|f\left(x\right)\big|\text{dx}$.
    \end{stav}
\end{stavbox}

\textit{Dokaz:} Kako važi $f\left(x\right) - |f\left(x\right)| \leq 0$, za svako $x\in \left[a, b\right]$, sledi $\displaystyle \int^b_a f\left(x\right)\text{dx} \leq \int^b_a |f\left(x\right)|\text{dx}$.
Takođe, važi $0\leq|f\left(x\right)| + f\left(x\right)$, za svako $x\in \left[a, b\right]$, odakle je $\displaystyle -\int^b_a f\left(x\right)\text{dx} \leq \int^b_a |f\left(x\right)|\text{dx}$.
Dakle, $\displaystyle \bigg|\int^b_a f\left(x\right)\text{dx}\bigg| \leq \int^b_a\big|f\left(x\right)\big|\text{dx}$.
\null\hfill $\blacksquare$\par

\begin{stavbox}
    \label{stav_2.10}
    \begin{stav}
        Neka su $f, g \in R\left[a,b\right]$ i neka za svako $x \in \left[a, b\right]$ važi $f\left(x\right) \leq g\left(x\right)$. Tada je $$\displaystyle \int^b_a f\left(x\right) \text{dx} \leq \int^b_a g\left(x\right)\text{dx}.$$
    \end{stav}
\end{stavbox}

\textit{Dokaz:} Primetimo da je $g-f \in R\left[a, b\right]$ i da je $\left(g-f\right)\left(x\right) \geq 0$, za svako $x\in\left[a,b\right]$. Tada je po
\hyperref[stav_2.8]{stavu 2.8} $\displaystyle \int^b_a \left(g-f\right)\left(x\right)\text{dx} \geq 0,$
odakle dobijamo $\displaystyle \int^b_a g\left(x\right)\text{dx} \geq  \int^b_a f\left(x\right)\text{dx} $.
\null\hfill $\blacksquare$\par

\begin{stavbox}
    \label{stav_2.11}
    \begin{stav}
        Neka je $f \in R\left[a, b\right]$ i neka su $\displaystyle m = \inf_{x\in \left[a, b\right]} f\left(x\right)$ i $\displaystyle M = \sup_{x \in \left[a,b\right]} f\left(x\right)$. Tada postoji $\mu \in \left[m, M\right]$ takvo da je $\displaystyle \int^b_a f\left(x\right) \text{dx} = \mu\left(b-a\right)$.
    \end{stav}
\end{stavbox}

\textit{Dokaz:} Za svako $x \in \left[a, b\right]$ važi:
\begin{align*}
    m \leq                 & f\left(x\right) \leq M                                             \\
    m\left(b-a\right) \leq & \int^b_a f\left(x\right)\text{dx} \leq M\left(b-a\right)           \\
    m \leq                 & \frac{1}{\left(b-a\right)}\int^b_a f\left(x\right)\text{dx} \leq M
\end{align*}
Dakle, postoji $\mu\in\left[m, M\right]$, takvo da je $\displaystyle \mu=\frac{1}{b-a}\int^b_a f\left(x\right)\text{dx}$.
\null\hfill $\blacksquare$\par

\begin{teoremabox}
    \label{podsetnik_teoreme_3}
    \textbf{Podsetnik Vajerštrasove teoreme:} Ako je funkcija neprekidna na intervalu $\left[a,b\right]$, ona je ograničena i
    dostiže svoj maksimum i minimum.
\end{teoremabox}

\begin{stavbox}
    \label{stav_2.12}
    \begin{stav}
        Neka je $f: \left[a, b\right] \longrightarrow \mathbb{R}$ neprekidna funkcija na $\left[a, b\right]$. Tada postoji $c\in \left[a,b\right]$ takvo da je
        $$ \int^b_a f\left(x\right)\text{dx} = f\left(c\right)\left(b-a\right).$$
    \end{stav}
\end{stavbox}

\textit{Dokaz:} Kako je funkcija $f$ neprekidna na $\left[a, b\right]$ po Vajerštrasovoj teoremi važi da je $\displaystyle \inf_{x\in\left[a,b\right]} f\left(x\right) = \min_{x \in \left[a, b\right]} f\left(x\right)$ i
$\displaystyle \sup_{x\in\left[a,b\right]} f\left(x\right) = \max_{x \in \left[a, b\right] f\left(x\right)}f\left(x\right)$. Neka su $\displaystyle m = \min_{x\in\left[a,b\right]} f\left(x\right)$ i  $\displaystyle M = \max_{x\in\left[a,b\right]} f\left(x\right)$.
Na osnovu prethodnog stava sledi da postoji $\mu \in \left[m, M\right]$ takvo da je $\displaystyle \int^b_a f\left(x\right) \text{dx} = \mu \left(b-a\right)$. Kako je funkcija $f$ neprekidna
na $\left[a, b\right]$ sledi da je $f\left(\left[a,b\right]\right) = \left[m, M\right]$. Otuda za $\mu \in \left[m, M\right]$ postoji $c \in \left[a, b\right]$ takvo da je $f\left(c\right) = \mu$.
\null\hfill $\blacksquare$\par

\subsection{Veza određenog integrala i izvoda. Njutn-Lajbnicova formula}

\begin{defbox}
    \label{definicija_2.13}
    \begin{definicija}
        Neka je $f \in R\left[a, b\right]$. Ima smisla razmatrati funkciju $\phi:\left[a, b\right] \longrightarrow \mathbb{R}$ definisanu sa $\phi \left(x\right) = \displaystyle \int^x_a f\left(t\right)\text{dt}$. Tada funkciju
        $\phi$ nazivamo \textbf{integral sa promenljivom gornjom granicom}.
    \end{definicija}
\end{defbox}

\begin{teoremabox}
    \label{teorema_2.7}
    \begin{teorema}
        Funkcija $\phi$ je neprekidna na $\left[a, b\right]$.
    \end{teorema}
\end{teoremabox}

\textit{Dokaz:} Neka je $x_0 \in \left[a ,b\right]$ proizvoljno.
Dokažimo da je funkcija $\phi$ neprekidna u $x_0$.
Neka je $M = \displaystyle \sup_{x\in\left[a,b\right]} | f\left(x\right) |$ i neka je $\varepsilon > 0$
proizvoljno. Tada za $x\in\left[a,b\right]$ važi:
\begin{equation*}
    \displaystyle |\phi\left(x\right) - \phi\left(x_0\right)| = \bigg|\int^x_a f\left(t\right)\text{dt} - \int^{x_0}_a f\left(t\right)\text{dt}\bigg| = \bigg|\int^x_{x_0}f\left(t\right)\text{dt}\bigg| \leq \bigg|\int^x_{x_0}|f\left(t\right)|\text{dt}\bigg| \leq M|x - x_0|
\end{equation*}
Neka je $\delta=\frac{\varepsilon}{M}$. Tada na osnovu $\varepsilon - \delta$ definicije važi
$$    \displaystyle \left(\forall \varepsilon > 0\right)\left(\exists \delta > 0\right)\left(\forall x\in \left[a, b\right]\right)\left(|x-x_0| < \delta \longrightarrow |\phi\left(x\right) - \phi\left(x_0\right)| < \varepsilon\right),$$
čime smo dokazali da je $\phi$ neprekidna u proizvoljnoj tački, zbog čega je neprekidna i na celom intervalu.
\null\hfill $\blacksquare$\par

\begin{teoremabox}
    \label{teorema_2.8}
    \begin{teorema}
        Ako je funkcija $f$ neprekidna na $\left[a, b\right]$ onda za funkciju $\phi$
        važi da je diferencijabilna na $\left(a,b\right)$ i važi $\varphi_+'\left(a\right) = f\left(a\right)$ i $\phi'_-\left(b\right) = f\left(b\right)$.
    \end{teorema}
\end{teoremabox}

\textit{Dokaz:} Neka je $x \in \left[a, b\right]$ proizvoljno i $h \in \mathbb{R}$,
takvo da je $x+h\in\left[a,b\right]$. Tada je
$$    \frac{\phi\left(x+h\right) - \phi\left(x\right)}{h}  = \frac{1}{h} \left( \int^{x+h}_a f\left(t\right)\text{dt} - \int^x_a f\left(t\right)\text{dt}\right) = \frac{1}{h} \int^{x+h}_x f\left(t\right)\text{dt}.$$
Po \hyperref[stav_2.12]{stavu 2.12} postoji $c\in\left[\min\left\{x, x+h\right\}, \max\left\{x, x+h\right\}\right]$ takvo da
$\displaystyle\int^{x+h}_x f\left(t\right)\text{dt}=f\left(c\right)\left(x+h-x\right)$.
Neka je $c=x+h\theta\left(x,h\right)$, $\theta\in\left[0,1\right]$. Tada je
$$ \frac{1}{h} \int^{x+h}_x f\left(t\right)\text{dt}=\frac{1}{h} f\left(x+h\theta\left(x,h\right)\right)\left(x+h-x\right)=f\left(x+h\theta\left(x,h\right)\right).$$
Odatle dobijamo $\displaystyle \phi'\left(x\right) =\lim\limits_{h\longrightarrow 0}\frac{\phi\left(x+h\right) - \phi\left(x\right)}{h} = \lim\limits_{h\longrightarrow 0} f\left(x+h\theta\left(x, h\right)\right)= f\left(x\right)$.
Slično se dokazuje i za $\phi'_+\left(a\right)$ i $\phi'_-\left(b\right)$.
\null\hfill $\blacksquare$\par

\begin{tvrbox}
    \label{tvrđenje_2.1}
    \begin{tvr}
        Neka je $f: \left(a, b\right) \longrightarrow \mathbb{R}$ neprekidna funkcija na $\left(a,b\right)$. Tada funkcija $f$ ima primitivnu funkciju $F: \left(a,b\right)\longrightarrow\mathbb{R}$ na $\left(a, b\right)$.
    \end{tvr}
\end{tvrbox}
\textit{Dokaz:} Neka je $x_0\in\left(a, b\right)$ fiksirana tačka i neka je $F:\left(a, b\right) \longrightarrow \mathbb{R}$ definisana sa $F\left(x\right) = \displaystyle\int^x_{x_0} f\left(t\right)\text{dt}$.
Funkcija $F$ je korektno definisana, jer je $f$ neprekidna funkcija na $\left[\min\left\{x, x_0\right\}, \max\left\{x, x_0\right\}\right]\subset\left(a, b\right)$. Neka
je $x\in\left(a, b\right)$ proizvoljno i neka je $h\in\mathbb{R}$ takvo da je $x+h\in\left(a, b\right)$. Tada analogono dokazu prethodne teoreme dobijamo
$\displaystyle\frac{F\left(x+h\right)-F\left(x\right)}{h} = \frac{1}{h}\int^{x+h}_x f\left(t\right)\text{dt}$.
Analogno prethodnoj teoremi dobijamo $f\left(x+h\theta\left(x, h\right)\right)$, gde je $\theta\left(x, h\right)\in\left[0,1\right]$.
Otuda je $\displaystyle \lim\limits_{h\longrightarrow 0}\frac{F\left(x+h\right)-F\left(x\right)}{h} = \lim\limits_{h\longrightarrow 0} f\left(x+h\theta\left(x, h\right)\right) = f\left(x\right)$, tačnije
$F'\left(x\right) = f\left(x\right)$.
\null\hfill $\blacksquare$\par

\subsubsection{Uraditi}
\begin{primbox}
    \label{primer_2.4}
    \begin{prim}
        Navesti primer funkcije koja nije neprekidna, ali ima primitivnu funkciju.
    \end{prim}
    $f\left(x\right) = \begin{cases}
            2x\sin\frac{1}{x} - \cos\frac{1}{x}, & x\in\left(-1,0\right)\cup\left(0,1\right) \\
            0,                                   & x = 0
        \end{cases}$
\end{primbox}

\begin{teoremabox}
    \label{teorema_2.9}
    \begin{teorema}
        (Njutn-Lajbnicova formula) Neka je $f:\left[a, b\right]\longrightarrow \mathbb{R}$ neprekidna funkcija na $\left[a,b\right]$ i neka je $F:\left[a, b\right]\longrightarrow\mathbb{R}$ primitivna funkcija funkcije f na $\left[a,b\right]$. Pri čemu važi $F'_+ \left(a\right) = f\left(a\right)\,\,\,F'_-\left(b\right) = f\left(b\right)$. Tada je
        $$ \int^b_a f\left(x\right)\text{dx} = F\left(b\right) - F\left(a\right).$$
    \end{teorema}
\end{teoremabox}

\textit{Dokaz:} Neka je $\phi: \left[a, b\right] \longrightarrow \mathbb{R}$ definisana sa
$\phi\left(x\right) = \displaystyle\int^x_a f\left(t\right)\text{dt}$. Na osnovu prethodnih teorema znamo da za svako $x\in \left[a, b\right]$ važi $\phi'\left(x\right) = f\left(x\right)$,
tačnije, $\phi$ je primitivna funkcija funkcije $f$. Stoga, postoji
$C \in \mathbb{R}$, takvo da za svako $x \in \left[a, b\right]$ važi
$\phi\left(x\right) = F\left(x\right) + C$ i $\displaystyle\phi\left(a\right)=\int_{a}^{a} f\left(t\right)\text{dt}=0$. Sledi da je
\begin{align*}
    \displaystyle\int^b_a f\left(t\right)\text{dt} & = \phi\left(b\right) = F\left(b\right) + C = F\left(b\right) +C+ \phi\left(a\right) - \phi\left(a\right) \\
                                                   & = F\left(b\right) +C+ \phi\left(a\right) - F\left(a\right)-C = F\left(b\right) - F\left(a\right).
\end{align*}
\null\hfill $\blacksquare$\par

\subsection{Smena promenljive i parcijalna integracija u određenom integralu}

\begin{teoremabox}
    \label{teorema_2.10}
    \begin{teorema}
        (Teorema o smeni promenljive) Neka je $f:\left[a, b\right]\longrightarrow \mathbb{R}$ neprekidna funkcija na $\left[a,b\right]$ i neka je $\phi: \left[\alpha, \beta\right] \longrightarrow \left[a, b\right]$ neprekidna funkcija na $\left[\alpha, \beta\right]$. Tada je:
        $$\displaystyle \int^{\phi\left(\beta\right)}_{\phi\left(\alpha\right)} f\left(x\right)\text{dx} = \int^\beta_\alpha \left(\left(f\circ\phi\right)\phi'\right)\left(t\right)\text{dt}$$
    \end{teorema}
\end{teoremabox}

\textit{Dokaz:} Neka je $F:\left[a,b\right]\longrightarrow \mathbb{R}$ primitivna funkcija funkcije $f:\left[a,b\right]\longrightarrow\mathbb{R}$. Tada je po Njutn-Lajbnicovoj formuli:
\begin{equation}
    \label{teorema_2.10:ek1}
    \displaystyle\int^{\phi\left(\beta\right)}_{\phi\left(\alpha\right)}f\left(x\right)\text{dx} = F\left(\phi\left(\beta\right)\right)-F\left(\phi\left(\alpha\right)\right)
\end{equation}
S druge strane za funkciju $\psi:\left[\alpha,\beta\right]\longrightarrow\mathbb{R}$ definisanu sa $\psi\left(t\right) = F\left(\phi\left(t\right)\right)$ važi:
\begin{equation*}
    \psi'\left(t\right) = F'\left(\phi\left(t\right)\right)\phi'\left(t\right) = \left(\left(f\circ\phi\right)\phi'\right)\left(t\right)
\end{equation*}
Dakle, funkcija $\psi$ je primitivna funkcija funkcije $\left(f\circ\phi\right)\phi'$ na intervalu $\left[\alpha,\beta\right]$, odakle važi:
\begin{equation}
    \displaystyle\int^\beta_\alpha \left(\left(f\circ\phi\right)\phi'\right)\left(t\right)\text{dt} = \psi\left(\beta\right) - \psi\left(\alpha\right) = F\left(\phi\left(\beta\right)\right) - F\left(\phi\left(\alpha\right)\right)
    \label{teorema_2.10:ek2}
\end{equation}
Iz \eqref{teorema_2.10:ek1} i \eqref{teorema_2.10:ek2} dobijamo $\displaystyle\int^{\phi\left(\beta\right)}_{\phi\left(\alpha\right)} f\left(x\right)\text{dx} = \int^\beta_\alpha \left(\left(f\circ\phi\right)\phi'\right)\left(t\right)\text{dt}$ što je kraj dokaza.
\null\hfill $\blacksquare$\par

\begin{teoremabox}
    \label{teorema_2.11}
    \begin{teorema}
        (Teorema o parcijalnoj integraciji) Neka su $u, v: \left[a, b\right] \longrightarrow\mathbb{R}$ neprekidno diferencijabilne funkcije na $\left[a,b\right]$. Tada je:
        $$\displaystyle\int^b_a\left(uv'\right)\left(x\right)\text{dx} = \left(uv\right)\left(b\right) - \left(uv\right)\left(a\right) - \int^b_a\left(u'v\right)\left(x\right)\text{dx}$$
    \end{teorema}
\end{teoremabox}

\textit{Dokaz:} Po Njutn-Lajbnicovoj formuli važi:
\begin{equation}
    \label{teorema_2.9:ek1}
    \displaystyle\int^b_a \left(uv\right)'\left(x\right)\text{dx} = \left(uv\right)\left(b\right) - \left(uv\right)\left(a\right)
\end{equation}
Zbog linearnosti određenih integrala važi:
\begin{equation}
    \label{teorema_2.9:ek2}
    \displaystyle\int^b_a \left(uv\right)'\left(x\right)\text{dx} = \int^b_a\left(u'v + uv'\right)\left(x\right)\text{dx} = \int^b_a\left(u'v\right)\left(x\right)\text{dx} + \int^b_a\left(uv'\right)\left(x\right)\text{dx}
\end{equation}
Iz jednakosti \eqref{teorema_2.9:ek1} i \eqref{teorema_2.9:ek2} važi:
$$\int^b_a\left(uv'\right)\left(x\right)\text{dx}=\displaystyle\int^b_a \left(uv\right)'\left(x\right)\text{dx} - \int^b_a\left(u'v\right)\left(x\right)\text{dx}= \left(uv\right)\left(b\right) - \left(uv\right)\left(a\right) - \int^b_a\left(u'v\right)\left(x\right)\text{dx}$$
\null\hfill $\blacksquare$\par

\end{document}